\documentclass[12pt]{ctexart}
\usepackage[a4paper,margin=2.5cm]{geometry}
\usepackage{hyperref}
\usepackage{enumitem}
\usepackage{booktabs}
\usepackage{verbatim}
\usepackage{xcolor}

\hypersetup{
  colorlinks=true,
  linkcolor=blue,
  urlcolor=cyan
}

\title{提示工程实践教程}
\author{工具箱系列}
\date{\today}

\begin{document}
\maketitle

\section{简介}
提示工程是引导大型语言模型 (LLM) 产生高质量输出的系统化方法。优秀的提示设计能够显著提升模型的可靠性、事实性与可控度, 让我们在写作助手、数据分析、自动化客服等场景中快速构建可复用的智能能力。

在实际项目中, 提示工程扮演以下角色:
\begin{itemize}[leftmargin=*,itemsep=0.4em]
  \item \textbf{桥梁}: 让业务需求与模型能力产生精确映射。
  \item \textbf{控制台}: 通过约束语气、格式与推理路径, 控制模型输出的结构。
  \item \textbf{调优器}: 结合同步记录与回测机制, 持续改进提示并沉淀最佳实践。
\end{itemize}

\section{入门}
全面掌握提示工程的第一步是搭建稳定的实验环境, 随后通过小规模试验认识模型的响应规律。

\subsection{环境搭建}
建议按照表~\ref{tab:setup} 准备基础设施, 以便快速开展实验。

\begin{table}[h]
  \centering
  \caption{常见环境组件及建议}
  \label{tab:setup}
  \begin{tabular}{p{4cm}p{9cm}}
    \toprule
    组件 & 要点 \\
    \midrule
    模型与接口 & 申请 OpenAI、Azure OpenAI、Anthropic 或本地开源模型, 记录密钥与速率限制。 \\
    开发工具 & 使用 VS Code、Cursor 等编辑器, 安装 Python/JavaScript SDK 与 API 调用脚本。 \\
    版本管理 & 使用 Git 管理提示与实验脚本, 通过分支记录迭代。 \\
    观察与记录 & 建立 Markdown/Jupyter 文档, 保留提示、响应、指标与反思。 \\
    团队协作 & 使用共享仓库或知识库整理最佳实践, 便于复用。 \\
    \bottomrule
  \end{tabular}
\end{table}

将环境搭建过程固化为自动化脚本 (如 Makefile、PowerShell 或 Python 工具), 可以降低新成员接入成本。

\subsection{小试牛刀}
完成基础设置后, 通过最小可行任务感受 LLM 的行为。以下步骤在实践中十分高效:

\begin{enumerate}[leftmargin=*,itemsep=0.5em]
  \item \textbf{设定角色}: 将模型置于具体场景, 如 ``你是一名资深技术写作者''。
  \item \textbf{描述目标}: 明确输入来源、任务边界与输出格式。
  \item \textbf{提供范例}: 给出 1--2 个理想回答示例, 帮助模型学习风格。
  \item \textbf{记录观察}: 从准确度、完整性、语气与潜在幻觉四个角度评价结果。
\end{enumerate}

示例:\par
\begin{verbatim}
系统提示: 你是一名资深法律顾问, 擅长用简明方式解释合同条款。
用户输入: 请用不超过 120 字 summarise 下列条款的主要责任和免责点: ...
期望输出: 1. 主要责任; 2. 免责条件; 3. 风险提示。
\end{verbatim}

通过更改语气、增加或删除示例、调整输出结构, 可以观察模型响应的差异, 逐步理解其敏感点。

\section{指南}
本章节依据 \href{https://prompt-engineering.xiniushu.com/}{Prompt Engineering Guide} 汇总常见提示模式, 并给出针对性的示例。

\subsection{提示原则}
设计提示时应确保信息充分、表达清晰、结构稳定。

\begin{enumerate}[leftmargin=*,itemsep=0.4em]
  \item \textbf{明确目标}: 说明任务、受众与输出质量标准。
  \item \textbf{化繁为简}: 对复杂任务拆分步骤, 用编号或小标题控制逻辑顺序。
  \item \textbf{锚定事实}: 将关键数据或引用直接嵌入提示, 提醒模型引用来源。
  \item \textbf{设置约束}: 例如 ``仅输出 JSON''、``避免主观评价'' 等强约束描述。
\end{enumerate}

示例:\par
\begin{verbatim}
任务: 分析安全事件报告并输出 JSON。
要求:
1. 识别根因, 需引用原文证据。
2. 按 {"impact":[],"root_causes":[],"actions":[]} 返回。
3. 若缺少信息, 使用 null 占位并说明原因。
\end{verbatim}

\subsection{如何迭代优化}
优质提示往往经过多轮验证与精炼。

\begin{enumerate}[leftmargin=*,itemsep=0.4em]
  \item \textbf{建立回放集}: 为关键任务收集典型输入, 形成 ``提示测试集''。
  \item \textbf{单变量试验}: 每次只调整语气、结构或示例之一, 对比差异。
  \item \textbf{量化指标}: 设计可检验的成功标准, 如正确字段数、事实准确率。
  \item \textbf{自动回归}: 使用脚本批量调用模型, 记录每版提示的指标变化。
\end{enumerate}

迭代日志示例:\par
\begin{verbatim}
v0 提示: 输出要点但缺少引用 -> 调整要求中加入 "引用原文"。
v1 提示: 引用过长 -> 在要求中限制引用不超过 30 字。
v2 提示: 结果稳定, 进入回归测试集。
\end{verbatim}

\subsection{文本总结}
总结任务关注范围限定与写作角度。

设计思路:
\begin{itemize}[leftmargin=*,itemsep=0.4em]
  \item 指定摘要长度、目标读者与应包含的重点。
  \item 建议提供示范摘要, 用于传达语气与排版要求。
  \item 要求模型列出未覆盖的信息, 方便后续补充。
\end{itemize}

示例提示:\par
\begin{verbatim}
请阅读下列产品更新日志, 生成 150 字内的中文摘要, 面向产品经理:
- 强调对用户体验的影响
- 以条目形式列出
- 指出仍待验证的假设
原始日志: [...]
\end{verbatim}

\subsection{文本推断}
推断任务常见于情感分析、意图识别与标签提取。

关键操作:
\begin{itemize}[leftmargin=*,itemsep=0.4em]
  \item 提供标签定义与互斥关系。
  \item 要求模型引用原文证据段落。
  \item 对不确定情况设置 ``无法判断'' 选项。
\end{itemize}

示例:\par
\begin{verbatim}
请判断评论的情绪标签, 仅可选择 {"正面","中性","负面","不确定"}。
输出 JSON, 包含字段:
- label: 标签
- evidence: 引用原文句子
- confidence: 0-1 之间的小数
评论: [...]
\end{verbatim}

\subsection{文本转换}
转换任务强调格式控制与语义保持。

建议:
\begin{itemize}[leftmargin=*,itemsep=0.4em]
  \item 明确输入输出示例, 特别是 JSON 键名、表格列名等。
  \item 定义错误处理策略, 例如缺少字段时输出 ``missing''。
  \item 在提示中提醒模型不要添加额外解释文字。
\end{itemize}

示例:\par
\begin{verbatim}
将下列 CSV 行转换为 JSON Lines。
输入示例:
name,email,role
Zhang Yan,zhang@example.com,Account Manager
输出要求: 每行一个 JSON 对象, 字段 {"name","email","role"}。
\end{verbatim}

\subsection{文本扩展}
扩展任务需要在创造力与一致性之间取得平衡。

实践要点:
\begin{itemize}[leftmargin=*,itemsep=0.4em]
  \item 明确扩展方向: 补充细节、拓展论证、延伸情节等。
  \item 设置语气、读者层次与篇幅上限。
  \item 要求模型指出新增内容与原文之间的联系, 防止偏离主题。
\end{itemize}

示例:\par
\begin{verbatim}
请将以下段落扩展为约 300 字, 面向高校新生, 语气鼓励:
- 保持原始观点不变
- 每个新增段落以主题句开头
原文: [...]
\end{verbatim}

\subsection{聊天机器人}
设计聊天机器人需要构建角色、上下文记忆与安全策略。

操作清单:
\begin{itemize}[leftmargin=*,itemsep=0.4em]
  \item \textbf{人格}: 描述背景、口吻与知识边界。
  \item \textbf{对话状态}: 说明如何总结历史对话, 何时向外部工具提问。
  \item \textbf{拒绝策略}: 给出处理敏感或越权请求的流程。
  \item \textbf{记忆管理}: 规定合并旧对话或丢弃冗余信息的规则。
\end{itemize}

片段提示示例:\par
\begin{verbatim}
系统: 你是某银行的智能客服 Lucy, 只能回答个人储蓄相关问题。
- 若问题涉及贷款或投资, 请礼貌拒绝并推荐人工客服。
- 若用户超过 3 轮未提供关键信息, 请总结目前信息并提出澄清问题。
用户: 我想了解信用卡增值服务...
\end{verbatim}

\subsection{总结}
成功的提示工程流程可概括为 ``设定目标 → 结构化表达 → 迭代优化 → 回归验证''。

为确保团队能力持续积累, 建议:
\begin{itemize}[leftmargin=*,itemsep=0.4em]
  \item 建立提示模板库与案例库, 支持跨项目复用。
  \item 持续维护指标仪表板, 监控提示质量与模型版本变化。
  \item 推动知识分享, 让团队形成共同的审核与改进标准。
\end{itemize}

\section*{参考资源}
\begin{itemize}[leftmargin=*,itemsep=0.4em]
  \item \href{https://prompt-engineering.xiniushu.com/}{Prompt Engineering Guide}
  \item \href{https://platform.openai.com/docs/guides/prompt-engineering}{OpenAI 官方提示工程指南}
  \item \href{https://www.promptingguide.ai/}{Prompting Guide}
\end{itemize}

\end{document}
