\documentclass[11pt]{article}
\usepackage[margin=1in]{geometry}
\usepackage{amsmath,amssymb,amsthm,bm}
\usepackage{hyperref}
\usepackage{graphicx}
\usepackage{caption}
\usepackage{booktabs}
\usepackage{listings}
\usepackage{xcolor}
\usepackage{float}
\usepackage{placeins}
\usepackage{enumitem}
\usepackage{subcaption}

% Unified graphics path (not required here but standard in project)
% \graphicspath{{figures/}}

% Unified listings style (used across chapters)
\lstdefinestyle{code}{%
  language=Python,
  basicstyle=\ttfamily\small,
  numbers=left,
  numberstyle=\tiny,
  keywordstyle=\color{blue}\bfseries,
  commentstyle=\color{teal!70!black},
  stringstyle=\color{orange!70!black},
  breaklines=true,
  frame=single,
  rulecolor=\color{black!30},
  tabsize=2,
  showstringspaces=false
}
\lstset{style=code}
\setlist{nosep, leftmargin=2em}

\title{LaTeX Basics: A Practical Guide}
\author{}
\date{\today}

\begin{document}
\maketitle
\tableofcontents
\clearpage

\section{Introduction}
LaTeX is a typesetting system designed for high-quality technical documents. This guide covers the core syntax and conventions used in this project so you can write and compile chapters efficiently.

\section{Install and Compile}
Install a TeX distribution (TeX Live/MacTeX/MiKTeX). Prefer XeLaTeX for Unicode support.
\begin{itemize}
  \item English: \texttt{xelatex your\_file.tex}
  \item Chinese: use \texttt{ctexart} class with \texttt{xelatex} or \texttt{lualatex}
\end{itemize}

\subsection{Engines}
\begin{itemize}
  \item \textbf{pdfLaTeX:} classic and stable; limited for CJK; avoid for CN.
  \item \textbf{XeLaTeX (recommended):} Unicode and system fonts; great for mixed languages.
  \item \textbf{LuaLaTeX:} modern and extensible; similar CN friendliness.
\end{itemize}

\subsection{Compilation Tips}
\begin{itemize}
  \item Compile twice for cross-references to resolve.
  \item Use UTF-8 source encoding.
  \item Install missing packages via TeX Live Manager/MiKTeX Console.
\end{itemize}

\section{Document Skeleton}
Minimal English document preamble used in this repository:
\begin{lstlisting}[style=code,caption={Minimal preamble (EN)}]
\documentclass[11pt]{article}
\usepackage[margin=1in]{geometry}
\usepackage{amsmath,amssymb,amsthm,bm}
\usepackage{hyperref,graphicx,caption,listings,xcolor,float,placeins}
\graphicspath{{figures/}} % images under figures/
\end{lstlisting}

For Chinese documents use \texttt{ctexart}:
\begin{lstlisting}[style=code,caption={Minimal preamble (CN)}]
\documentclass[UTF8,zihao=-4]{ctexart}
\usepackage[a4paper,margin=2.5cm]{geometry}
\usepackage{amsmath,amssymb,amsthm,bm}
\usepackage{hyperref,graphicx,caption,listings,xcolor,float,placeins}
\graphicspath{{figures/}}
\end{lstlisting}

\section{Text, Sections, and Lists}
\subsection{Headings and Styles}
Use sectioning commands and basic text formatting:
\begin{lstlisting}[style=code,caption={Text and lists}]
\section{Title} \subsection{Subtitle} \subsubsection{Subsubtitle}
Bold: \textbf{...}, italics: \emph{...}, monospace: \texttt{...}
\begin{itemize} \item bullet A \item bullet B \end{itemize}
\begin{enumerate} \item first \item second \end{enumerate}
\end{lstlisting}

\section{Math}
Inline math uses \$...\$, display math uses \texttt{equation}/\texttt{align}:
\begin{lstlisting}[style=code,caption={Math examples}]
Inline: $a^2 + b^2 = c^2$.
\begin{equation}
  \sigma(z) = \frac{1}{1+e^{-z}}
\end{equation}
\begin{align}
  J(\bm{w}) &= \frac{1}{2}\lVert \bm{w} \rVert^2 + C \sum_i \xi_i \\
            &\text{s.t. } y_i(\bm{w}^\top \bm{x}_i + b) \ge 1 - \xi_i
\end{align}
Matrix: $\begin{bmatrix} a & b \\ c & d \end{bmatrix}$
\end{lstlisting}

\subsection{Quick Structures}
\begin{itemize}
  \item Fractions/sums/norms: \verb|\frac{a}{b}|, \verb|\sum_{i=1}^n|, \verb|\lVert x \rVert_2|.
  \item Expectation/variance: \verb|\mathbb{E}[X]|, \verb|\operatorname{Var}(X)|; Gaussian: \verb|\mathcal{N}(\mu,\Sigma)|.
  \item Align: use \verb|&| for alignment points and \verb|\\| for new lines.
\end{itemize}

\section{Figures}
Keep images in a local \texttt{figures/} folder and reference via \texttt{\graphicspath}. Typical figure block:
\begin{lstlisting}[style=code,caption={Figure environment}]
\begin{figure}[H]
  \centering
  % \includegraphics[width=0.9\linewidth]{example.png}
  \caption{Caption text}
  \label{fig:example}
\end{figure}
\FloatBarrier
\end{lstlisting}

\section{Tables}
\begin{lstlisting}[style=code,caption={Table environment}]
\begin{table}[H]
  \centering
  \caption{Results} \label{tab:res}
  \begin{tabular}{lcc}
    \toprule
    Method & Acc & F1 \\
    \midrule
    A & 0.90 & 0.88 \\
    B & 0.92 & 0.90 \\
    \bottomrule
  \end{tabular}
\end{table}
\FloatBarrier
\end{lstlisting}

\subsection{Alignment and Width}
Use \verb|l/c/r| for text alignment; \verb|p{4cm}| for fixed-width columns with wrapping.

\section{Code Listings}
Configure \texttt{listings} once in the preamble, then:
\begin{lstlisting}[style=code,caption={Listing usage}]
\begin{lstlisting}[style=code,caption={Title},label={lst:ex}]
# your code here
\end{lstlisting}
% Or include a file
\lstinputlisting[style=code,caption={script.py},label={lst:py}]{script.py}
\end{lstlisting}

\section{Cross-References and Links}
Label then reference: \verb|\label{fig:ex}| and refer with \verb|Figure~\ref{fig:ex}|. URLs via \verb|\url{https://example.com}| from \texttt{hyperref}.

\section{Theorems and Definitions}
Define environments using \texttt{amsthm}:
\begin{lstlisting}[style=code,caption={Theorem environment}]
\newtheorem{theorem}{Theorem}
\begin{theorem}
  Every finite tree has a leaf.
\end{theorem}
\end{lstlisting}

\noindent Similarly define \verb|definition|, \verb|lemma|, etc., and combine with \verb|\label|/\verb|\ref| for cross-referencing.

\section{Common Macros}
Use \verb|\newcommand| for reuse:
\begin{lstlisting}[style=code,caption={Macros}]
\newcommand{\vect}[1]{\bm{#1}} % bold vector
\newcommand{\E}{\mathbb{E}}   % expectation
\end{lstlisting}

\section{Layout Tips}
\begin{itemize}
\subsection{Common Packages}
\begin{itemize}
  \item \texttt{amsmath, amssymb, amsthm}: math and theorem environments.
  \item \texttt{bm}: bold math symbols (e.g., \verb|\bm{x}|).
  \item \texttt{graphicx}: images; \verb|\includegraphics|.
  \item \texttt{caption, float, placeins}: captions and float placement (\verb|[H]|, \verb|\FloatBarrier|).
  \item \texttt{listings}: code blocks; unified style in preamble.
  \item \texttt{booktabs}: beautiful tables (\verb|\toprule|/\verb|\midrule|/\verb|\bottomrule|).
  \item \texttt{subcaption}: subfigures (\verb|subfigure| environment).
\end{itemize}
  \item \textbf{Figures/Tables:} place \verb|\caption| before \verb|\label|; use \verb|[H]| and \verb|\FloatBarrier| to control float placement.
\subsection{Paragraphs and Spacing}
Chinese docs indent by default; for extra paragraph spacing consider \verb|\setlength{\parskip}{.5em}| (use sparingly).
  \item \textbf{Overfull boxes:} allow hyphenation or tweak spacing; URLs auto-wrap with \verb|\url|.
  \item \textbf{Consistency:} keep preamble and styles unified across chapters.
\end{itemize}

\section{Quick Reference}
\paragraph{Math snippets} $\frac{a}{b}$, $\sum_{i=1}^n$, $\nabla f$, $\mathcal{N}(\mu,\Sigma)$, $\operatorname*{argmin}\limits_x f(x)$.

\subsection{Subfigures}
\begin{lstlisting}[style=code,caption={Subfigures (needs subcaption)}]
\begin{figure}[H]
  \centering
  \begin{subfigure}[t]{0.48\linewidth}
    % \includegraphics[width=\linewidth]{fig_a.png}
    \caption{A}
  \end{subfigure}\hfill
  \begin{subfigure}[t]{0.48\linewidth}
    % \includegraphics[width=\linewidth]{fig_b.png}
    \caption{B}
  \end{subfigure}
  \caption{Subfigure demo}
  \label{fig:subfig}
\end{figure}
\end{lstlisting}
\paragraph{Figure/table flow} Always provide \verb|\caption| then \verb|\label|; use \verb|[H]| and \verb|\FloatBarrier| to control placement.

\paragraph{Compilation tips} Run XeLaTeX twice to resolve references; keep sources in UTF-8.

\section{Troubleshooting}
\begin{itemize}
  \item Missing package: install via TeX Live Manager/MiKTeX Console.
  \item Figures not found: check \verb|\graphicspath| and filenames.
  \item Unicode issues: compile with XeLaTeX and save as UTF-8.
\end{itemize}

\appendix
\section{Appendix: Handy Syntax Cheats}
\begin{itemize}
  \item Inline/display: \verb|$...$|, \verb|\[ ... \]|; multiline alignment via \verb|align|.
  \item Symbols: \verb|\mathbb{R}|, \verb|\mathcal{L}|, \verb|\bm{x}|, \verb|\langle x,y \rangle|.
  \item Floats: \verb|figure/table| with \verb|\caption+\label|; use \verb|[H]| and \verb|\FloatBarrier|.
  \item Code: \verb|listings| \verb|\lstlisting| and \verb|\lstinputlisting|.
\end{itemize}

\end{document}
