% 建议使用 XeLaTeX 或 LuaLaTeX 编译(更佳的中文支持)
\documentclass[UTF8,zihao=-4]{ctexart}

% 统一导言(与项目章节保持一致)
\usepackage[a4paper,margin=2.5cm]{geometry}
\usepackage{amsmath,amssymb,amsthm}
\usepackage{bm}
\usepackage{hyperref}
\usepackage{graphicx}
\usepackage{caption}
\usepackage{listings}
\usepackage{xcolor}
\usepackage{float}
\usepackage{placeins}

% \graphicspath{{figures/}} % 本教程无需插图,保持注释

% 统一代码风格(与项目一致)
\lstdefinestyle{code}{%
  language=Python,
  basicstyle=\ttfamily\small,
  numbers=left,
  numberstyle=\tiny,
  keywordstyle=\color{blue}\bfseries,
  commentstyle=\color{teal!70!black},
  stringstyle=\color{orange!70!black},
  breaklines=true,
  frame=single,
  rulecolor=\color{black!30},
  tabsize=2,
  showstringspaces=false
}
\lstset{style=code}

\title{LaTeX 基础教程:项目文档速成}
\author{}
\date{\today}

\begin{document}
\maketitle

\section{引言}
LaTeX 是面向高质量排版的文档系统,特别适合学术论文与技术文档。本教程聚焦本项目常用语法与约定,帮助你快速完成章节撰写与编译。

\section{安装与编译}
安装 TeX Live/MacTeX/MiKTeX,推荐使用 XeLaTeX 以获得稳定的 Unicode 支持:
\begin{itemize}
  \item 英文文档:\texttt{xelatex your\_file.tex}
  \item 中文文档:\texttt{ctexart} 文档类配合 \texttt{xelatex}/\texttt{lualatex}
\end{itemize}

\section{文档骨架}
项目英文章节的最小导言:
\begin{lstlisting}[style=code,caption={最小导言(英文)}]
\documentclass[11pt]{article}
\usepackage[margin=1in]{geometry}
\usepackage{amsmath,amssymb,amsthm,bm}
\usepackage{hyperref,graphicx,caption,listings,xcolor,float,placeins}
\graphicspath{{figures/}}
\end{lstlisting}

中文章节使用 \texttt{ctexart}:
\begin{lstlisting}[style=code,caption={最小导言(中文)}]
\documentclass[UTF8,zihao=-4]{ctexart}
\usepackage[a4paper,margin=2.5cm]{geometry}
\usepackage{amsmath,amssymb,amsthm,bm}
\usepackage{hyperref,graphicx,caption,listings,xcolor,float,placeins}
\graphicspath{{figures/}}
\end{lstlisting}

\section{正文、分节与列表}
\begin{lstlisting}[style=code,caption={文本与列表}]
\section{标题} \subsection{小节} \subsubsection{子小节}
加粗:\textbf{...},斜体:\emph{...},等宽:\texttt{...}
\begin{itemize} \item 条目 A \item 条目 B \end{itemize}
\begin{enumerate} \item 第一 \item 第二 \end{enumerate}
\end{lstlisting}

\section{数学公式}
行内由 \$...\$ 包围,陈列公式用 \texttt{equation}/\texttt{align}:
\begin{lstlisting}[style=code,caption={数学示例}]
行内:$a^2 + b^2 = c^2$。
\begin{equation}
  \sigma(z) = \frac{1}{1+e^{-z}}
\end{equation}
\begin{align}
  J(\bm{w}) &= \frac{1}{2}\lVert \bm{w} \rVert^2 + C \sum_i \xi_i \\
            &\text{s.t. } y_i(\bm{w}^\top \bm{x}_i + b) \ge 1 - \xi_i
\end{align}
矩阵:$\begin{bmatrix} a & b \\ c & d \end{bmatrix}$
\end{lstlisting}

\section{插图}
图片统一放在本章的 \texttt{figures/} 下,通过 \texttt{\graphicspath} 管理路径:
\begin{lstlisting}[style=code,caption={插图环境}]
\begin{figure}[H]
  \centering
  % \includegraphics[width=0.9\linewidth]{example.png}
  \caption{图注}
  \label{fig:example}
\end{figure}
\FloatBarrier
\end{lstlisting}

\section{表格}
\begin{lstlisting}[style=code,caption={表格环境}]
\begin{table}[H]
  \centering
  \caption{结果} \label{tab:res}
  \begin{tabular}{lcc}
    \hline
    方法 & Acc & F1 \\
    \hline
    A & 0.90 & 0.88 \\
    B & 0.92 & 0.90 \\
    \hline
  \end{tabular}
\end{table}
\FloatBarrier
\end{lstlisting}

\section{代码示例(listings)}
在导言区配置一次,正文中使用:
\begin{lstlisting}[style=code,caption={listings 用法}]
\begin{lstlisting}[style=code,caption={标题},label={lst:ex}]
# your code here
\end{lstlisting}
% 或者直接纳入文件
\lstinputlisting[style=code,caption={script.py},label={lst:py}]{script.py}
\end{lstlisting}

\section{交叉引用与超链接}
先 \verb|\label{fig:ex}| 再在正文处 \verb|图~\ref{fig:ex}| 引用;网址用 \verb|\url{https://example.com}|(需 \texttt{hyperref})。

\section{定理环境}
用 \texttt{amsthm} 定义:
\begin{lstlisting}[style=code,caption={定理环境}]
\newtheorem{theorem}{Theorem}
\begin{theorem}
  Every finite tree has a leaf.
\end{theorem}
\end{lstlisting}

\section{常用宏命令}
\begin{lstlisting}[style=code,caption={宏命令}]
\newcommand{\vect}[1]{\bm{#1}} % 向量加粗
\newcommand{\E}{\mathbb{E}}   % 期望
\end{lstlisting}

\section{速查要点}
\paragraph{数学片段} $\frac{a}{b}$,$\sum_{i=1}^n$,$\nabla f$,$\mathcal{N}(\mu,\Sigma)$,$\operatorname*{argmin}\limits_x f(x)$。

\paragraph{图表规范} 先 \verb|\caption| 再 \verb|\label|;必要时用 \verb|[H]| 与 \verb|\FloatBarrier| 控制位置。

\paragraph{编译建议} XeLaTeX 编译;为了交叉引用生效可运行两遍;保存为 UTF-8。

\section{常见问题}
\begin{itemize}
  \item 缺少宏包:使用 TeX Live Manager/MiKTeX Console 安装。
  \item 找不到图片:检查 \verb|\graphicspath| 与文件名。
  \item 乱码:使用 XeLaTeX 并将源文件保存为 UTF-8。
\end{itemize}

\end{document}
% 建议使用 XeLaTeX 或 LuaLaTeX 编译(更佳的中文支持)
\documentclass[UTF8,zihao=-4]{ctexart}

% 统一导言(与项目章节保持一致)
\usepackage[a4paper,margin=2.5cm]{geometry}
\usepackage{amsmath,amssymb,amsthm}
\usepackage{bm}
\usepackage{hyperref}
\usepackage{graphicx}
\usepackage{caption}
\usepackage{booktabs}  % 更漂亮的表格线
\usepackage{listings}
\usepackage{xcolor}
\usepackage{float}
\usepackage{placeins}
\usepackage{enumitem}  % 列表间距与样式控制
\usepackage{subcaption} % 子图排版(可选)

% \graphicspath{{figures/}} % 本教程无需插图,保持注释

% 统一代码风格(与项目一致)
\lstdefinestyle{code}{%
  language=Python,
  basicstyle=\ttfamily\small,
  numbers=left,
  numberstyle=\tiny,
  keywordstyle=\color{blue}\bfseries,
  commentstyle=\color{teal!70!black},
  stringstyle=\color{orange!70!black},
  breaklines=true,
  frame=single,
  rulecolor=\color{black!30},
  tabsize=2,
  showstringspaces=false
}
\lstset{style=code}
% 列表格式更紧凑清晰
\setlist{nosep, leftmargin=2em}

\title{LaTeX 基础教程:项目文档速成}
\author{}
\date{\today}

\begin{document}
\maketitle
\tableofcontents
\clearpage

\section{引言}
LaTeX 是面向高质量排版的文档系统,特别适合学术论文与技术文档。本教程聚焦本项目常用语法与约定,帮助你快速完成章节撰写与编译,并提供可直接复制修改的代码片段。

\section{安装与编译}
安装 TeX Live/MacTeX/MiKTeX,推荐使用 XeLaTeX 以获得稳定的 Unicode 支持:
\begin{itemize}
  \item 英文文档:\texttt{xelatex your\_file.tex}
  \item 中文文档:\texttt{ctexart} 文档类配合 \texttt{xelatex}/\texttt{lualatex}
\end{itemize}

\subsection{常用引擎与何时使用}
\begin{itemize}
  \item \textbf{pdfLaTeX:} 传统、稳定,但中文支持不如 XeLaTeX;不建议用于本项目中文文档。
  \item \textbf{XeLaTeX(推荐):} 原生 Unicode 与系统字体支持,适合中英混排。
  \item \textbf{LuaLaTeX:} 功能强大,性能与可扩展性更好;与 XeLaTeX 类似,中文也友好。
\end{itemize}

\subsection{编译建议}
\begin{itemize}
  \item 交叉引用(图表/公式)需编译两次确保 \verb|\ref| 正常。
  \item 文档编码统一为 UTF-8。
  \item 缺包时用 TeX Live Manager 或 MiKTeX Console 安装。
\end{itemize}

\section{文档骨架}
项目英文章节的最小导言:
\begin{lstlisting}[style=code,caption={最小导言(英文)}]
\documentclass[11pt]{article}
\usepackage[margin=1in]{geometry}
\usepackage{amsmath,amssymb,amsthm,bm}
\usepackage{hyperref,graphicx,caption,listings,xcolor,float,placeins}
\graphicspath{{figures/}}
\end{lstlisting}

中文章节使用 \texttt{ctexart}:
\begin{lstlisting}[style=code,caption={最小导言(中文)}]
\documentclass[UTF8,zihao=-4]{ctexart}
\usepackage[a4paper,margin=2.5cm]{geometry}
\usepackage{amsmath,amssymb,amsthm,bm}
\usepackage{hyperref,graphicx,caption,listings,xcolor,float,placeins}
\graphicspath{{figures/}}
\end{lstlisting}

\subsection{常用宏包说明}
\begin{itemize}
  \item \texttt{amsmath, amssymb, amsthm}:数学环境与符号、定理环境。
  \item \texttt{bm}:粗体数学符号(如 \verb|\bm{x}|)。
  \item \texttt{graphicx}:插图;\verb|\includegraphics| 引入图片。
  \item \texttt{caption, float, placeins}:图表标题、浮动体与位置控制(\verb|[H]|、\verb|\FloatBarrier|)。
  \item \texttt{listings}:代码高亮与排版;本项目统一风格见导言。
  \item \texttt{booktabs}:优雅表格线(\verb|\toprule|/\verb|\midrule|/\verb|\bottomrule|)。
  \item \texttt{subcaption}:子图布局(\verb|subfigure| 环境)。
\end{itemize}

\section{正文、分节与列表}
\subsection{分节命令与文字样式}
\begin{lstlisting}[style=code,caption={文本与列表}]
\section{标题} \subsection{小节} \subsubsection{子小节}
加粗:\textbf{...},斜体:\emph{...},等宽:\texttt{...}
\begin{itemize} \item 条目 A \item 条目 B \end{itemize}
\begin{enumerate} \item 第一 \item 第二 \end{enumerate}
\end{lstlisting}

\subsection{段落与间距}
中文文档默认首行缩进;需要段间距可在导言设置 \verb|\setlength{\parskip}{.5em}|(谨慎使用)。

\section{数学公式}
行内由 \$...\$ 包围,陈列公式用 \texttt{equation}/\texttt{align}:
\begin{lstlisting}[style=code,caption={数学示例}]
行内:$a^2 + b^2 = c^2$。
\begin{equation}
  \sigma(z) = \frac{1}{1+e^{-z}}
\end{equation}
\begin{align}
  J(\bm{w}) &= \frac{1}{2}\lVert \bm{w} \rVert^2 + C \sum_i \xi_i \\
            &\text{s.t. } y_i(\bm{w}^\top \bm{x}_i + b) \ge 1 - \xi_i
\end{align}
矩阵:$\begin{bmatrix} a & b \\ c & d \end{bmatrix}$
\end{lstlisting}

\subsection{常用结构速查}
\begin{itemize}
  \item 分式:\verb|\frac{a}{b}|,求和:\verb|\sum_{i=1}^n|,范数:\verb|\lVert x \rVert_2|。
  \item 期望/方差:\verb|\mathbb{E}[X]|、\verb|\operatorname{Var}(X)|;高斯:\verb|\mathcal{N}(\mu,\Sigma)|。
  \item 对齐:\verb|align| 环境用 \verb|&| 对齐,行末 \verb|\\| 换行。
\end{itemize}

\section{插图}
图片统一放在本章的 \texttt{figures/} 下,通过 \texttt{\graphicspath} 管理路径:
\begin{lstlisting}[style=code,caption={插图环境}]
\begin{figure}[H]
  \centering
  % \includegraphics[width=0.9\linewidth]{example.png}
  \caption{图注}
  \label{fig:example}
\end{figure}
\FloatBarrier
\end{lstlisting}

\subsection{子图示例}
\begin{lstlisting}[style=code,caption={子图排版(需要 subcaption)}]
\begin{figure}[H]
  \centering
  \begin{subfigure}[t]{0.48\linewidth}
    \centering
    % \includegraphics[width=\linewidth]{fig_a.png}
    \caption{子图 A}
  \end{subfigure}\hfill
  \begin{subfigure}[t]{0.48\linewidth}
    \centering
    % \includegraphics[width=\linewidth]{fig_b.png}
    \caption{子图 B}
  \end{subfigure}
  \caption{子图示例}
  \label{fig:subfig}
\end{figure}
\end{lstlisting}

\section{表格}
\begin{lstlisting}[style=code,caption={表格环境}]
\begin{table}[H]
  \centering
  \caption{结果} \label{tab:res}
  \begin{tabular}{lcc}
    \toprule
    方法 & Acc & F1 \\
    \midrule
    A & 0.90 & 0.88 \\
    B & 0.92 & 0.90 \\
    \bottomrule
  \end{tabular}
\end{table}
\FloatBarrier
\end{lstlisting}

\subsection{对齐与格式}
\verb|l/c/r| 控制列左/中/右对齐;用 \verb|p{4cm}| 可指定列宽实现自动换行。

\section{代码示例(listings)}
在导言区配置一次,正文中使用:
\begin{lstlisting}[style=code,caption={listings 用法}]
\begin{lstlisting}[style=code,caption={标题},label={lst:ex}]
# your code here
\end{lstlisting}
% 或者直接纳入文件
\lstinputlisting[style=code,caption={script.py},label={lst:py}]{script.py}
\end{lstlisting}

\section{交叉引用与超链接}
先 \verb|\label{fig:ex}| 再在正文处 \verb|图~\ref{fig:ex}| 引用;网址用 \verb|\url{https://example.com}|(需 \texttt{hyperref})。

\section{定理环境}
用 \texttt{amsthm} 定义:
\begin{lstlisting}[style=code,caption={定理环境}]
\newtheorem{theorem}{Theorem}
\begin{theorem}
  Every finite tree has a leaf.
\end{theorem}
\end{lstlisting}

\noindent 可类似定义 \verb|definition|、\verb|lemma| 等环境;使用时配合 \verb|\label| 与 \verb|\ref| 交叉引用。

\section{常用宏命令}
\begin{lstlisting}[style=code,caption={宏命令}]
\newcommand{\vect}[1]{\bm{#1}} % 向量加粗
\newcommand{\E}{\mathbb{E}}   % 期望
\end{lstlisting}

建议将复用的命令集中管理,避免章节间不一致。

\section{版式与排版建议}
\begin{itemize}
  \item \textbf{图表规范:} 先 \verb|\caption| 再 \verb|\label|;大量图后用 \verb|\FloatBarrier|,必要时给 \verb|figure/table| 加 \verb|[H]| 固定位置。
  \item \textbf{文本溢出:} 出现 overfull 警告时,适当断词(英语)或调整行间/段间距;网址可用 \verb|\url| 自动断行。
  \item \textbf{统一风格:} 与项目章节保持相同导言与代码风格,便于维护与审阅。
\end{itemize}

\section{速查要点}
\paragraph{数学片段} $\frac{a}{b}$,$\sum_{i=1}^n$,$\nabla f$,$\mathcal{N}(\mu,\Sigma)$,$\operatorname*{argmin}\limits_x f(x)$。

\paragraph{图表规范} 先 \verb|\caption| 再 \verb|\label|;必要时用 \verb|[H]| 与 \verb|\FloatBarrier| 控制位置。

\paragraph{编译建议} XeLaTeX 编译;为了交叉引用生效可运行两遍;保存为 UTF-8。

\section{常见问题}
\begin{itemize}
  \item 缺少宏包:使用 TeX Live Manager/MiKTeX Console 安装。
  \item 找不到图片:检查 \verb|\graphicspath| 与文件名。
  \item 乱码:使用 XeLaTeX 并将源文件保存为 UTF-8。
\end{itemize}

\appendix
\section{附录:常用语法速查清单}
\begin{itemize}
  \item 行内/陈列:\verb|$...$|,\verb|\[ ... \]|;多行对齐用 \verb|align|。
  \item 经典符号:\verb|\mathbb{R}|、\verb|\mathcal{L}|、\verb|\bm{x}|、\verb|\langle x,y \rangle|。
  \item 浮动体:\verb|figure/table| 搭配 \verb|\caption+\label|,必要时加 \verb|[H]| 与 \verb|\FloatBarrier|。
  \item 代码:\verb|listings| 的 \verb|\lstlisting| 与 \verb|\lstinputlisting|。
\end{itemize}

\end{document}
