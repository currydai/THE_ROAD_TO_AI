% 编译建议:使用 XeLaTeX 或 LuaLaTeX(中文支持更好)
\documentclass[UTF8,zihao=-4]{ctexart}

% 基本排版与超链接
\usepackage[a4paper,margin=2.5cm]{geometry}
\usepackage{hyperref}
\hypersetup{
  colorlinks=true,
  linkcolor=blue,
  urlcolor=blue,
  citecolor=blue
}

% 代码高亮(不依赖外部编译器)
\usepackage{listings}
\usepackage{xcolor}
\lstdefinestyle{codestyle}{
  basicstyle=\ttfamily\small,
  numbers=left,
  numberstyle=\tiny,
  numbersep=8pt,
  keywordstyle=\color{blue},
  commentstyle=\color{teal!70!black},
  stringstyle=\color{orange!70!black},
  showstringspaces=false,
  breaklines=true,
  frame=single,
  framerule=0.3pt,
  rulecolor=\color{black!15}
}
\lstset{style=codestyle}

% 紧凑列表
\usepackage{enumitem}
\setlist[itemize]{nosep,leftmargin=1.2em}
\setlist[enumerate]{nosep,leftmargin=1.6em}

\title{Codex 中文使用指南:特性与新手上手}
\author{}
\date{\today}

\begin{document}

\maketitle
\tableofcontents
\vspace{0.5em}

\section{简介}
Codex 是面向开发者的智能编码助手,能够理解自然语言描述并生成、解释或改写代码,以提升开发效率和代码质量。它支持主流编程语言与工具链,可融入常见 IDE/编辑器与命令行工作流,帮助你更快地从问题到实现、从思路到代码。

\paragraph{核心价值}
用“对话式编程”替代部分机械工作:将需求、约束和上下文清晰表达给 Codex,让它产出符合风格与规范的代码草稿;你负责审阅、完善与集成,整体迭代更高效。

\section{主要特性}
\begin{itemize}
  \item \textbf{自然语言到代码}:根据描述自动生成函数、类、脚本或配置;支持指定语言、框架与风格。
  \item \textbf{多语言支持}:常见语言(如 Python、JavaScript/TypeScript、Java、C/C++、Go、Rust 等)与脚本/配置(如 Bash、SQL、YAML)。
  \item \textbf{上下文理解}:可结合现有代码片段、文件结构与接口约定,生成更贴合项目的实现与说明。
  \item \textbf{代码补全与重构建议}:给出补全、重构、性能优化与可读性改进思路。
  \item \textbf{注释与文档生成}:从代码反推注释、README 片段、使用示例与变更说明。
  \item \textbf{测试辅助}:生成单元测试样例、断言建议与边界用例提示。
  \item \textbf{调试与解释}:对报错与堆栈给出成因分析与可能的修复策略。
  \item \textbf{工作流集成}:可在 IDE、CLI 与 CI 环境中协同使用,融入现有分支、评审与发布流程。
\end{itemize}

\section{快速上手(面向新手)}
\subsection{准备工作}
\begin{enumerate}
  \item \textbf{选择运行环境}:推荐在常用 IDE(如 VS Code)或命令行环境中使用 Codex,以便与项目文件无缝协作。
  \item \textbf{配置访问凭据}:如需云端模型访问,按提示配置 API Key 或登录账号;确保网络与代理设置可用。
  \item \textbf{准备项目上下文}:打开项目根目录,确保依赖安装与脚本可运行,便于 Codex 理解与复现。
\end{enumerate}

\subsection{第一个示例:用描述生成函数}
在交互窗口(IDE/CLI)中给出明确需求、输入/输出与约束:

\begin{lstlisting}[language=bash,caption={以自然语言描述需求}]
# 目标:实现一个去重且保持顺序的函数
# 语言:Python;复杂度 O(n);需包含简单单元测试
\end{lstlisting}

Codex 将返回示例实现与测试骨架。检查是否满足要求,必要时追加“请将集合替换为 OrderedDict”“添加边界用例”等二次指令进行迭代。

\subsection{让 Codex 解释与修复}
\begin{lstlisting}[language=bash,caption={让 Codex 解释报错并提出修复}]
# 我遇到错误:ValueError: unexpected shape (3,)
# 上下文:这是 numpy 矩阵运算;贴上相关代码片段
# 期望:解释原因,给出两种修复方案,并说明各自权衡
\end{lstlisting}

\subsection{生成文档与注释}
\begin{lstlisting}[language=bash,caption={从代码生成注释与 README 片段}]
# 任务:为下列函数补充中文注释与示例;
# 另生成 README 的“快速开始”与“API 概览”小节
\end{lstlisting}

\section{提示词(Prompt)编写建议}
\begin{itemize}
  \item \textbf{提供上下文}:说明项目背景、关键接口、输入/输出、边界条件与依赖版本。
  \item \textbf{明确约束}:指定语言/框架、复杂度/性能目标、风格规范(如 PEP 8)、是否需要测试与注释。
  \item \textbf{分步推进}:先生成草稿,再针对问题点逐步细化要求,避免一次性堆叠太多目标。
  \item \textbf{给出示例}:提供一小段期望风格或调用方式的样例,有助于对齐输出形式。
  \item \textbf{让 Codex 解释}:要求其输出设计思路与权衡,便于你做出取舍与复核。
\end{itemize}

\section{典型工作流示例}
\begin{enumerate}
  \item \textbf{理解需求}:用自然语言描述功能点、输入/输出、边界与验收标准。
  \item \textbf{生成实现}:让 Codex 给出初稿,要求附带注释与测试样例。
  \item \textbf{本地验证}:运行测试/脚本,记录失败用例与性能瓶颈。
  \item \textbf{迭代改进}:将失败现象和性能数据反馈给 Codex,请其优化或重构。
  \item \textbf{文档完善}:生成 README 片段与使用示例,统一风格并落库。
\end{enumerate}

\section{常见问题与排查}
\begin{itemize}
  \item \textbf{中文乱码}:优先使用 XeLaTeX/LuaLaTeX 编译(本指南即采用),或在工程中启用中文字体支持包(如 \texttt{ctex})。
  \item \textbf{依赖不一致}:在提示中标注运行环境(语言版本/依赖版本/操作系统),并让 Codex 给出对应安装脚本。
  \item \textbf{输出不符合规范}:在提示中明确代码风格与静态检查规则,并要求 Codex 对照修正。
  \item \textbf{结果不稳定}:将任务拆小、固定输入/输出示例;必要时请 Codex 解释其选择与假设。
\end{itemize}

\section{安全、合规与隐私}
\begin{itemize}
  \item \textbf{最小化敏感信息}:避免在提示中包含密钥、凭证、个人隐私或未脱敏样本。
  \item \textbf{开源合规}:当 Codex 参考开源实现时,务必检查许可证与归属要求。
  \item \textbf{人工复核}:对生成代码进行静态检查、测试与代码评审,确保质量与安全。
\end{itemize}

\section{与开发流程的集成建议}
\begin{itemize}
  \item \textbf{与 IDE 协作}:在代码旁边与 Codex 对话,直接引用当前文件/选中片段作为上下文。
  \item \textbf{与 CLI 协作}:用命令行驱动批量任务(如批量生成测试、批量重构建议)。
  \item \textbf{与 CI 协作}:在合并前自动运行测试与静态检查;必要时让 Codex 依据失败日志生成修复建议。
\end{itemize}

\section{编译与阅读说明}
\begin{itemize}
  \item \textbf{编译方式}:建议用 XeLaTeX 或 LuaLaTeX 编译,以获得更好的中文与字体支持。
  \item \textbf{文件结构}:本指南为单文件文档 \texttt{tools/codex/codex.tex},可按需拆分为章节。
  \item \textbf{可复用片段}:将“提示词模板”“工作流清单”“排查清单”提炼为团队共用文档,提升一致性。
\end{itemize}

\end{document}
