% 建议使用 XeLaTeX 或 LuaLaTeX 编译(中文与公式支持更佳)
\documentclass[UTF8,zihao=-4]{ctexart}

% 版式与常用宏包(与章节保持一致风格)
\usepackage[a4paper,margin=2.5cm]{geometry}
\usepackage{amsmath, amssymb, amsthm}
\usepackage{bm}
\usepackage{hyperref}
\usepackage{graphicx}
\usepackage{caption}
\usepackage{float}
\usepackage{placeins}
\usepackage{listings}
\usepackage{xcolor}
\graphicspath{{figures/}}

% 代码样式(与教程统一)
\lstdefinestyle{code}{
  basicstyle=\ttfamily\small,
  numbers=left,
  numberstyle=\tiny,
  numbersep=8pt,
  keywordstyle=\color{blue},
  commentstyle=\color{teal!70!black},
  stringstyle=\color{orange!70!black},
  showstringspaces=false,
  breaklines=true,
  frame=single,
  framerule=0.3pt,
  rulecolor=\color{black!15}
}
\lstset{style=code}

% 常用快捷记号(不额外引入包)
\newcommand{\R}{\mathbb{R}}
\newcommand{\N}{\mathbb{N}}
\newcommand{\Z}{\mathbb{Z}}
\newcommand{\Q}{\mathbb{Q}}
\newcommand{\C}{\mathbb{C}}
\newcommand{\E}{\mathbb{E}}
\newcommand{\Var}{\mathrm{Var}}
\newcommand{\Cov}{\mathrm{Cov}}
\newcommand{\Prob}{\mathbb{P}}
\newcommand{\KL}{\mathrm{D}_{\mathrm{KL}}}
\newcommand{\tr}{\mathrm{tr}}
\newcommand{\rank}{\mathrm{rank}}
\newcommand{\diag}{\mathrm{diag}}
\newcommand{\sign}{\mathrm{sign}}
\newcommand{\softmax}{\mathrm{softmax}}
\newcommand{\1}{\mathbf{1}}
\newcommand{\0}{\mathbf{0}}
\newcommand{\indep}{\perp\!\!\!\perp}

\title{机器学习常用数学符号与记号规范(Cheat Sheet)}
\author{}
\date{\today}

\begin{document}
\maketitle
\tableofcontents

\section{总览与风格}
为便于在全书/各章快速理解公式,本文件约定统一的符号风格:
\begin{itemize}
  \item \textbf{标量(scalar)}:斜体小写,如 $a, b, c, \alpha, \beta$;常数可用非斜体 $\mathrm{e},\ \pi$。
  \item \textbf{向量(vector)}:粗体小写,如 $\bm{x},\ \bm{w}$;分量 $x_i$。
  \item \textbf{矩阵(matrix)}:粗体大写,如 $\bm{X},\ \bm{W}$;元素 $[\bm{X}]_{ij}$、第 $i$ 行 $\bm{X}_{i,:}$、第 $j$ 列 $\bm{X}_{:,j}$。
  \item \textbf{张量(tensor)}:粗体花体或黑体,如 $\bm{\mathcal{X}}$(本书中常直接以粗体大写表示)。
  \item \textbf{集合/空间}:花体大写,如 $\mathcal{X},\ \mathcal{Y}$;数集用黑板粗体 $\R,\ \N,\ \Z$。
  \item \textbf{随机变量}:大写 $X$,其取值小写 $x$;向量同理 $\bm{X}$ 与 $\bm{x}$。
  \item \textbf{参数/模型}:参数向量 $\bm{\theta}$、权重 $\bm{w}$、偏置 $b$;模型 $f(\bm{x};\bm{\theta})$。
  \item \textbf{数据维度}:样本数 $n$、特征维度 $d$、类别数 $K$。
\end{itemize}

\section{线性代数与索引}
\begin{itemize}
  \item 单位矩阵 $\bm{I}_d$;全零向量 $\0$;全一向量 $\1$。
  \item 转置 $\bm{X}^\top$;逆 $\bm{X}^{-1}$(若存在);伪逆 $\bm{X}^{\dagger}$。
  \item 迹 $\tr(\bm{X})$;秩 $\rank(\bm{X})$;\(\diag(\bm{v})\) 生成对角矩阵。
  \item 内积 $\langle \bm{x},\bm{y} \rangle = \bm{x}^\top\bm{y}$;Hadamard 乘积 $\bm{X}\odot\bm{Y}$;Kronecker 乘积 $\bm{X}\otimes\bm{Y}$。
  \item 特征分解 $\bm{X}=\bm{Q}\,\diag(\bm{\lambda})\,\bm{Q}^\top$;奇异值分解 $\bm{X}=\bm{U}\,\diag(\bm{\sigma})\,\bm{V}^\top$。
  \item 切片与索引:标量 $x_i$、列向量 $\bm{X}_{:,j}$、行向量 $\bm{X}_{i,:}$。
\end{itemize}

\section{范数、距离与相似度}
\begin{itemize}
  \item 向量 $\ell_p$ 范数:$\lVert\bm{x}\rVert_p=\big(\sum_i |x_i|^p\big)^{1/p}$;常见:$\lVert\bm{x}\rVert_1,\ \lVert\bm{x}\rVert_2,\ \lVert\bm{x}\rVert_\infty$。
  \item 矩阵范数:Frobenius $\lVert\bm{X}\rVert_F=\sqrt{\sum_{ij} X_{ij}^2}$;谱范数 $\lVert\bm{X}\rVert_2=\sigma_{\max}(\bm{X})$。
  \item 欧氏距离 $\lVert\bm{x}-\bm{y}\rVert_2$;曼哈顿距离 $\lVert\bm{x}-\bm{y}\rVert_1$;马氏距离 $\sqrt{(\bm{x}-\bm{\mu})^\top\bm{\Sigma}^{-1}(\bm{x}-\bm{\mu})}$。
  \item 余弦相似度 $\dfrac{\langle\bm{x},\bm{y}\rangle}{\lVert\bm{x}\rVert_2\,\lVert\bm{y}\rVert_2}$。
\end{itemize}

\section{微积分与优化记号}
\begin{itemize}
  \item 梯度 $\nabla f(\bm{x})\in\R^d$;雅可比 $\bm{J}_f(\bm{x})$;Hessian $\nabla^2 f(\bm{x})$。
  \item 偏导 $\tfrac{\partial f}{\partial x_i}$;全微分 $\mathrm{d}f$;链式法则与向量化求导遵循矩阵维度一致性。
  \item 极值算子:$\arg\min_{\bm{\theta}} f(\bm{\theta})$、$\arg\max$;约束优化用拉格朗日函数 $\mathcal{L}(\bm{x},\bm{\lambda})$ 与 KKT 条件。
  \item 常见更新:梯度下降 $\bm{\theta}\leftarrow\bm{\theta}-\eta\,\nabla f(\bm{\theta})$;动量/Adam 超参 $\beta_1,\beta_2,\epsilon$。
\end{itemize}

\section{概率与统计记号}
\begin{itemize}
  \item 概率与密度:$\Prob(A)$,$p_X(x)$(密度/质量函数),条件 $p(y\mid x)$,联合 $p(x,y)$。
  \item 期望 $\E[Z]$;方差 $\Var(Z)$;协方差 $\Cov(X,Y)$;相关系数 $\rho_{XY}$。
  \item 独立/条件独立:$X\indep Y$,$X\indep Y\mid Z$。
  \item 大小符号:$\sim$ 表“服从”:$X\sim\mathcal{N}(\mu,\Sigma)$。
\end{itemize}

\subsection{常见分布}
\begin{itemize}
  \item 高斯:$\mathcal{N}(\mu,\sigma^2)$、多元 $\mathcal{N}(\bm{\mu},\bm{\Sigma})$。
  \item 伯努利/二项/多项:$\mathrm{Ber}(p)$、$\mathrm{Bin}(n,p)$、$\mathrm{Mult}(n,\bm{\pi})$;类别:$\mathrm{Cat}(\bm{\pi})$。
  \item 均匀:$\mathcal{U}(a,b)$;指数:$\mathrm{Exp}(\lambda)$;Gamma:$\mathrm{Ga}(\alpha,\beta)$;Beta:$\mathrm{Be}(\alpha,\beta)$;Dirichlet:$\mathrm{Dir}(\bm{\alpha})$。
\end{itemize}

\subsection{信息论}
\begin{itemize}
  \item 熵:$H(X)=-\sum_x p(x)\log p(x)$;条件熵 $H(Y\mid X)$;互信息 $I(X;Y)$。
  \item KL 散度:$\KL(p\Vert q)=\sum_x p(x)\log\dfrac{p(x)}{q(x)}$;交叉熵 $H(p,q)$。
\end{itemize}

\section{数据与模型记号}
\begin{itemize}
  \item 数据集:$\mathcal{D}=\{(\bm{x}_i,y_i)\}_{i=1}^n$;$\bm{x}_i\in\R^d$,$y_i\in\mathcal{Y}$(回归 $\subseteq\R$,分类 $\in\{1,\dots,K\}$)。
  \item 设计矩阵与标签向量:$\bm{X}\in\R^{n\times d}$,第 $i$ 行为 $\bm{x}_i^\top$;$\bm{y}\in\R^n$ 或 one-hot $\bm{Y}\in\{0,1\}^{n\times K}$。
  \item 模型:$f(\bm{x};\bm{\theta})$;参数 $\bm{\theta}$;预测 $\hat{y}$ 或 $\hat{\bm{y}}$。
  \item 损失:单样本 $\ell(\hat{y},y)$;经验风险 $\hat{R}(\bm{\theta})=\tfrac{1}{n}\sum_{i=1}^n \ell\big(f(\bm{x}_i;\bm{\theta}),y_i\big)$。
  \item 正则与目标:$\Omega(\bm{\theta})$(如 $\lambda\lVert\bm{w}\rVert_2^2$);总体目标 $J(\bm{\theta})=\hat{R}+\Omega$。
\end{itemize}

\section{分类与回归常用函数}
\begin{itemize}
  \item Sigmoid:$\sigma(t)=1/(1+e^{-t})$;Softmax:$\softmax(\bm{z})_k=\dfrac{e^{z_k}}{\sum_j e^{z_j}}$。
  \item 线性回归:$\hat{y}=\bm{w}^\top\bm{x}+b$;L2 损失:$\tfrac{1}{2}(\hat{y}-y)^2$。
  \item 逻辑回归:$p(y=1\mid\bm{x})=\sigma(\bm{w}^\top\bm{x}+b)$;对数似然与交叉熵如各章所述。
  \item SVM 间隔:$\gamma = y\,(\bm{w}^\top\bm{x}+b)/\lVert\bm{w}\rVert_2$;Hinge 损失:$\max(0,1-y\,f(\bm{x}))$。
\end{itemize}

\section{深度学习记号}
\begin{itemize}
  \item 层与维度:第 $\ell$ 层权重 $\bm{W}^{(\ell)}$、偏置 $\bm{b}^{(\ell)}$;预激活 $\bm{z}^{(\ell)}$、激活 $\bm{a}^{(\ell)}$。
  \item 小批量:批大小 $B$;样本索引集合 $\mathcal{B}$。
  \item 反向传播:$\dfrac{\partial\mathcal{L}}{\partial \bm{W}^{(\ell)}}$、$\dfrac{\partial\mathcal{L}}{\partial \bm{b}^{(\ell)}}$;链式法则逐层传递。
  \item 常见激活:ReLU $\max(0,t)$、Leaky-ReLU、Tanh、Sigmoid;归一化/正则化:BatchNorm、Dropout(保留率 $p$)。
\end{itemize}

\section{评估与度量}
\begin{itemize}
  \item 二分类:TP、TN、FP、FN;精准率 $\mathrm{Precision}=\tfrac{\mathrm{TP}}{\mathrm{TP}+\mathrm{FP}}$,召回率 $\mathrm{Recall}=\tfrac{\mathrm{TP}}{\mathrm{TP}+\mathrm{FN}}$,F1 $=2\tfrac{PR}{P+R}$。
  \item ROC/PR:TPR、FPR;AUC。
  \item 回归:MSE、MAE、$R^2$。
  \item 交叉验证:$K$ 折;训练/验证/测试划分:$\mathcal{D}_{\text{train}}/\mathcal{D}_{\text{val}}/\mathcal{D}_{\text{test}}$。
\end{itemize}

\section{集合与逻辑记号}
\begin{itemize}
  \item 基本运算:并 $A\cup B$、交 $A\cap B$、差 $A\setminus B$、补 $A^c$;基数 $|A|$。
  \item 指示函数:$\mathbf{1}_A(x)$,若 $x\in A$ 则为 1 否则为 0。
  \item 映射与函数:$f:\mathcal{X}\to\mathcal{Y}$;复合 $f\circ g$;恒等映射 $\mathrm{id}$。
\end{itemize}

\section{速查表(选)}
\begin{center}
\begin{tabular}{ll}
\hline
记号 & 含义 \\
\hline
$n, d, K$ & 样本数、特征维、类别数 \\
$\bm{x}\in\R^d$ & 特征向量 \\
$\bm{X}\in\R^{n\times d}$ & 设计矩阵(行:样本;列:特征) \\
$y\in\R$ 或 $\{1,\dots,K\}$ & 标签(回归/分类) \\
$f(\bm{x};\bm{\theta})$ & 模型 \\
$\ell(\hat{y},y)$ & 单样本损失 \\
$\hat{R}(\bm{\theta})$ & 经验风险 \\
$\Omega(\bm{\theta})$ & 正则项 \\
$J(\bm{\theta})$ & 目标函数 $\hat{R}+\Omega$ \\
$\E[\cdot],\ \Var,\ \Cov$ & 期望、方差、协方差 \\
$\KL(p\Vert q)$ & KL 散度 \\
$\tr(\cdot)$, $\rank(\cdot)$ & 迹、秩 \\
$\lVert\cdot\rVert_1,\ \lVert\cdot\rVert_2$ & 常用范数 \\
\hline
\end{tabular}
\end{center}

\section{备注}
不同书籍/论文偶有记号差异。本文件以“可读性与工程一致性”为先:
\begin{itemize}
  \item 粗体表示向量/矩阵,尽量避免与随机变量大小写规则冲突;
  \item 概率分布用 $p(\cdot)$ 或特定记号(如 $\mathcal{N}$、$\mathrm{Ber}$)清晰区分;
  \item 若章节有特别说明,以章节内局部约定为准,但建议与本表保持一致。
\end{itemize}

\end{document}

