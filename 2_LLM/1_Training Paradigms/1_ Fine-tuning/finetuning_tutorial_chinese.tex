\documentclass[UTF8,zihao=-4]{ctexart}
\usepackage[a4paper,margin=2.5cm]{geometry}
\usepackage{amsmath, amssymb, amsthm}
\usepackage{bm}
\usepackage{hyperref}
\usepackage{graphicx}
\usepackage{caption}
\usepackage{listings}
\usepackage{xcolor}
\usepackage{float}
\usepackage{placeins}
\graphicspath{{figures/}}

% 代码样式
\lstdefinestyle{code}{
  basicstyle=\ttfamily\small,
  numbers=left,
  numberstyle=\tiny,
  numbersep=8pt,
  keywordstyle=\color{blue},
  commentstyle=\color{teal!70!black},
  stringstyle=\color{orange!70!black},
  showstringspaces=false,
  breaklines=true,
  frame=single,
  framerule=0.3pt,
  rulecolor=\color{black!15}
}
\lstset{style=code}

\title{大模型微调范式:监督、指令与参数高效策略深度解析}
\author{}
\date{\today}

\begin{document}
\maketitle

\section{监督微调(Supervised Fine-Tuning, SFT)}
\subsection{目标定义与损失设计}
监督微调旨在将预训练语言模型适配到具体任务,通过少量标注样本更新模型参数。常见设置包括:
\begin{itemize}
  \item \textbf{条件生成任务:} 例如摘要、翻译、代码生成,采用自回归交叉熵损失,保持与预训练期望一致。
  \item \textbf{分类与检索:} 加入任务特定的线性头或提示模板,将输出序列映射为标签概率或稠密向量。
  \item \textbf{多任务联合:} 采用加权损失或多头架构,在同一模型中兼容问答、理由生成与工具调用。
\end{itemize}
针对小数据集,常使用标签平滑、Mixout、R-Drop 等正则化方法缓解过拟合。实践中还需规划适当的学习率和 warmup,以防止对预训练知识的灾难性遗忘。

\subsection{数据工程与样本构建}
高质量的标注数据直接决定 SFT 效果:
\begin{itemize}
  \item \textbf{任务分析:} 明确输入模式(问题、上下文、历史对话)与输出格式(简短答案、逐步推理),制定模板。
  \item \textbf{多阶段标注:} 先由模型自动生成候选,再由人类审校;结合双人标注和仲裁流程衡量一致性。
  \item \textbf{样本增强:} 通过 paraphrase、反事实编辑、知识扩写,提高数据多样性;保持难例与简单样本的平衡。
\end{itemize}
数据治理环节需要记录标注质量、审核意见与版本历史,以便对微调后的行为进行责任追溯。

\subsection{训练流程与评估指标}
SFT 通常采用较小 batch size 与较短训练轮次。关键实践包括:
\begin{itemize}
  \item \textbf{分层学习率:} 对底层 Transformer 施加较小学习率,对新加入的分类头使用更大更新比例。
  \item \textbf{梯度裁剪:} 控制梯度范数,防止少量噪声样本导致发散。
  \item \textbf{持续评估:} 监测任务准确率、BLEU/ROUGE、困惑度,同时进行人工抽检,评估可解释性、事实性与安全性。
\end{itemize}

\section{指令微调(Instruction Tuning, Chat Tuning)}
\subsection{指令语料构建与对齐维度}
指令微调通过引入多样化任务指令与示例对话,使模型学会遵循人类意图。数据来源包括 FLAN、Self-Instruct、OpenOrca 等。关键覆盖维度:
\begin{itemize}
  \item \textbf{任务多样性:} 包含分类、抽取、生成、推理、工具调用、数学等,以提升模型泛化能力。
  \item \textbf{角色与语气:} 指令中添加“你是一名导师/律师/医生”等角色设定,帮助模型适应不同上下文。
  \item \textbf{对齐层级:} 覆盖直接回答、追问澄清、拒答敏感需求、解释推理过程等行为模式。
\end{itemize}

\subsection{自监督扩展与半自动标注}
为了降低人力成本,常采用模型辅助生成 instruction 数据:
\begin{itemize}
  \item \textbf{Self-Instruct:} 预训练模型根据 seed 指令自动扩展新任务,再由人类筛选与修订。
  \item \textbf{反事实指令:} 生成要求模型拒绝或给出警告的场景,提升安全性。
  \item \textbf{难例挖掘:} 通过评估模型失误案例,生成针对性对话补充。
\end{itemize}
需要设计质量控制指标,如平均响应长度、引用来源比例、拒答准确率,确保数据覆盖真实应用需求。

\subsection{聊天调优与多轮上下文管理}
Chat Tuning 专注于多轮对话场景:
\begin{itemize}
  \item \textbf{系统指令与记忆:} 在上下文前置系统消息定义行为准则,并维护有限历史窗口或外部记忆模块。
  \item \textbf{对话状态压缩:} 使用摘要或关键词提取的方式,将长对话压缩为短上下文,降低推理成本。
  \item \textbf{拒答与安全策略:} 结合 RLHF 或 DPO,确保在越界请求时回复恰当的拒绝或引导。
\end{itemize}
评估时需模拟真实用户对话,关注连贯性、任务成功率与安全响应比率。

\section{Prompt 模板与系统角色(ChatML, Alpaca Format)}
\subsection{模板设计原则}
Prompt 模板决定模型的输入结构和上下文提示:
\begin{itemize}
  \item \textbf{结构化包裹:} 明确分隔系统、用户、助手角色,保证解析稳定;ChatML 使用 \texttt{<|system|>}、\texttt{<|user|>} 标签。
  \item \textbf{指令显式化:} 采用“Instruction + Input + Output”格式(Alpaca Format)强调任务说明与上下文。
  \item \textbf{约束与示例:} 在模板中嵌入风格说明、禁止事项、示例对话,减少模糊理解。
\end{itemize}

\subsection{模板适配与多任务共存}
面对多任务训练,需要设计可扩展模板体系:
\begin{itemize}
  \item \textbf{字段可选:} 对于无输入任务,提供空输入占位;对复杂任务,支持附加元数据如语言、长度限制。
  \item \textbf{自动化生成:} 通过模板渲染器将结构化数据映射为文本,避免人工拼接造成错误。
  \item \textbf{跨平台一致性:} 保证训练与线上推理使用同一模板,避免指令偏移。
\end{itemize}

\subsection{系统角色与安全护栏}
系统角色定义模型的全局行为边界:
\begin{itemize}
  \item \textbf{行为准则:} 说明模型价值观、优先级,如“遵守安全政策”“优先提供事实信息”。
  \item \textbf{工具调用:} 在系统消息中描述可用工具、调用格式与错误处理流程。
  \item \textbf{角色切换:} 为客服、写作助手、编程助手等预设不同系统消息,实现场景化响应。
\end{itemize}
设计系统消息时,要防止用户提示覆盖系统约束,可加入优先级声明与拒答策略。

\section{参数高效微调(PEFT:LoRA, QLoRA, Prefix-Tuning)}
\subsection{LoRA 与低秩适配}
LoRA 通过为权重矩阵添加低秩分解并仅训练低秩参数,实现显存友好的调优:
\begin{itemize}
  \item \textbf{原理:} 在权重 $W$ 上叠加 $BA$,其中 $A \in \mathbb{R}^{r \times d}$、$B \in \mathbb{R}^{k \times r}$,训练期间仅更新 $A,B$。
  \item \textbf{优势:} 显著降低参数量;支持快速切换任务特定 LoRA 模块;兼容混合精度。
  \item \textbf{部署:} 可将 LoRA 参数与主模型合并,或按需加载以支持多任务路由。
\end{itemize}

\subsection{QLoRA 与低比特量化}
QLoRA 在 LoRA 基础上结合 4-bit 量化,实现消费级 GPU 上的高效微调:
\begin{itemize}
  \item \textbf{量化策略:} 使用 NF4(Normalized Float 4)量化权重,保持幅度信息,结合 double quantization 减少量化误差。
  \item \textbf{优化器状态:} 通过 paged optimizer 将状态存储在 CPU 内存,配合分块加载缓解显存压力。
  \item \textbf{训练流程:} 采用 4-bit 权重 + 16-bit 激活的组合,梯度仍以 FP16/FP32 累积,保证稳定。
\end{itemize}

\subsection{Prefix/Prompt-Tuning 与可插拔控制}
前缀微调通过优化附加的连续提示向量实现任务适配:
\begin{itemize}
  \item \textbf{Prefix-Tuning:} 为每一层注意力引入可学习的前缀键值对,保持原始权重冻结。
  \item \textbf{P-Tuning v2:} 将可学习提示插入到 embedding 层,与虚拟 token 共享,适配更深层模型。
  \item \textbf{组合策略:} 将 Prefix 与 LoRA 结合,实现细粒度任务控制;或通过 Router 根据输入选择合适前缀。
\end{itemize}
PEFT 方法适合多任务部署,通过模块化加载实现快速切换。需要在评估阶段比较完整微调与 PEFT 的性能差距,并监测泛化与安全性。

\section*{参考文献}
\begin{itemize}
  \item Wei et al. ``Finetuned Language Models Are Zero-Shot Learners.'' ICLR, 2022.
  \item Chung et al. ``Scaling Instruction-Finetuned Language Models.'' arXiv, 2022.
  \item Dettmers et al. ``QLoRA: Efficient Finetuning of Quantized LLMs.'' arXiv, 2023.
  \item Hu et al. ``LoRA: Low-Rank Adaptation of Large Language Models.'' ICLR, 2022.
  \item Lester et al. ``The Power of Scale for Parameter-Efficient Prompt Tuning.'' EMNLP, 2021.
\end{itemize}

\end{document}

