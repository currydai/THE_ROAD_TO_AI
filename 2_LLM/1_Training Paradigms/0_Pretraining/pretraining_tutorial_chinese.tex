\documentclass[UTF8,zihao=-4]{ctexart}
\usepackage[a4paper,margin=2.5cm]{geometry}
\usepackage{amsmath, amssymb, amsthm}
\usepackage{bm}
\usepackage{hyperref}
\usepackage{graphicx}
\usepackage{caption}
\usepackage{listings}
\usepackage{xcolor}
\usepackage{float}
\usepackage{placeins}
\graphicspath{{figures/}}

% 代码样式
\lstdefinestyle{code}{
  basicstyle=\ttfamily\small,
  numbers=left,
  numberstyle=\tiny,
  numbersep=8pt,
  keywordstyle=\color{blue},
  commentstyle=\color{teal!70!black},
  stringstyle=\color{orange!70!black},
  showstringspaces=false,
  breaklines=true,
  frame=single,
  framerule=0.3pt,
  rulecolor=\color{black!15}
}
\lstset{style=code}

\title{大规模预训练范式:语料构建、数据治理与系统优化全景}
\author{}
\date{\today}

\begin{document}
\maketitle

\section{大规模语料构建(CC, The Pile, C4, RefinedWeb)}
\subsection{语料来源全景与覆盖维度}
预训练数据的多样性直接决定模型的语言覆盖范围、知识密度与风格稳健性。代表性开放语料集包括:
\begin{itemize}
  \item \textbf{Common Crawl (CC):} 每月抓取的万亿级网页集合,提供最广泛的主题覆盖,但噪声较大。CC 的子集如 CC-News、CC-Stories 为不同任务裁剪。
  \item \textbf{The Pile:} EleutherAI 构建的 825GB 高质量语料包,由 22 个子语料组成,涵盖学术论文、开源代码、问答论坛和文学作品,强调平衡性与高熵内容。
  \item \textbf{C4 (Colossal Clean Crawled Corpus):} 在 Common Crawl 基础上通过语言识别、重复检测与质量过滤得到的 750GB 英文语料,被 T5 等模型广泛采用。
  \item \textbf{RefinedWeb:} 针对 GPT-NeoX-20B 构建的 600GB 高质量网页语料,结合链接权重、URL 过滤和句子级评分以提升知识密度。
\end{itemize}
在实践中,现实的数据组合通常包含新闻、书籍、代码、对话、百科、社交媒体等多条数据链路,目标是在语言风格、领域知识、编程技能与跨模态描述之间形成平衡。

\subsection{语料扩增与地域语言覆盖}
构建全球可用的语料库需要兼顾多语种、方言与专业术语:
\begin{itemize}
  \item \textbf{跨语种爬取:} 通过站点白名单、语言检测和种子扩散策略获取中文、阿拉伯语、印地语等内容;结合本地新闻站、政府公文、维基百科镜像。
  \item \textbf{协作与众包:} 与高校、科研机构合作采集学术论文、教材、课程字幕;通过众包标注收集口语、方言数据。
  \item \textbf{开放数据接口:} 利用 arXiv、PubMed、StackExchange、GitHub Archive 等公开 API 持续更新专业语料。
\end{itemize}
在扩增过程中需关注版权与许可,确保数据遵循 CC 许可、GNU 文档许可或开源协议;对于未明确授权的数据需在合规范围内采样、去标识化或删除。

\subsection{分层存储与元数据管理}
大规模语料库应在获取阶段写入丰富的元数据以支持后续治理:
\begin{itemize}
  \item \textbf{文档级元数据:} 包含来源 URL、抓取时间、语言、字符计数、领域标签、许可证等信息,便于追踪与审计。
  \item \textbf{段落级特征:} 使用规则和模型生成毒性评分、可读性指数、主题分布、潜在版权风险标签。
  \item \textbf{版本与快照:} 对语料集进行快照管理,记录清洗和采样前后的差异,支持回滚和对比实验。
\end{itemize}
在工程上常用 HDFS、Delta Lake、Parquet 等格式分级存储,通过 Spark、Ray 或 Dask 批处理以满足 EB 级数据吞吐需求。

\section{数据清洗与去重(Deduplication, Filtering)}
\subsection{多级去重策略}
重复内容会放大训练偏差、浪费算力并导致过拟合。常见去重层次:
\begin{itemize}
  \item \textbf{URL 级去重:} 基于 URL 或域名的哈希集合快速过滤重复网页。
  \item \textbf{文档级去重:} 利用 SimHash、MinHash、LSH Forest 对文档指纹去重,控制语料的局部和全局重复率。
  \item \textbf{片段级去重:} 将文档切分为句子或段落,使用 n-gram 指纹或余弦相似度阈值过滤,防止大段重复。
\end{itemize}
针对学术论文或代码语料,还可配合 AST 抽象语法树匹配与引用网络分析,剔除镜像库或 fork 内容。

\subsection{质量过滤与安全治理}
清洗流程需要识别噪声、低质量或高风险文本:
\begin{itemize}
  \item \textbf{语言与字符检测:} 基于 FastText、CLD3 或字符分布检测非目标语言、乱码和爬虫痕迹。
  \item \textbf{毒性与偏见检测:} 使用 Detoxify、Perspective API、self-critique 模型打分,设定阈值或权重抽样,控制模型输出安全性。
  \item \textbf{隐私过滤:} 正则匹配与深度模型识别邮箱、身份证、电话号码等敏感字段,并通过模糊化或删除处理。
  \item \textbf{模板与广告识别:} 通过页面布局特征、HTML 标签和重复短语识别 SEO 垃圾页面,实现软黑名单。
\end{itemize}
在生产环境中,数据流水线通常由多级过滤模型串联构成 DAG,结合人类审核样本与反馈闭环持续优化阈值设置。

\subsection{加权采样与分布校准}
为避免模型被长尾领域或低质量数据污染,需要对清洗后的语料进行加权采样:
\begin{itemize}
  \item \textbf{领域分桶:} 依据主题或来源将语料划分为新闻、百科、论坛、代码、学术等分桶,设定目标分布并执行分层抽样。
  \item \textbf{温度采样:} 类似重排 softmax 的思想,对得分为 $s_i$ 的文档按照 $p_i = \frac{\exp(s_i/\tau)}{\sum_j \exp(s_j/\tau)}$ 加权,控制高分文档比例。
  \item \textbf{动态再加权:} 使用在线指标(如模型困惑度、梯度范数)监测训练进度,针对性增加困难样本或最新数据的采样权重。
\end{itemize}
在训练过程中,可通过数据缓冲队列和可复现的伪随机种子保证实验可重复性,并记录每次 epoch 的实际分布供分析。

\section{并行训练策略(Data / Model / Pipeline / Tensor)}
\subsection{数据并行:超大批次与同步协议}
数据并行通过复制完整模型到多个设备上独立处理不同 minibatch,再同步梯度。关键技术点包括:
\begin{itemize}
  \item \textbf{梯度压缩:} 使用 ZeRO、1-bit Adam、GossipGrad 等压缩方法降低通信带宽。
  \item \textbf{同步拓扑:} NCCL AllReduce、RingReduce、树形聚合;在多机多卡场景需要优化拓扑匹配和通信调度。
  \item \textbf{大批次训练:} 使用线性学习率预热与 LARS/LAMB 优化器维持收敛稳定性,可达 10M 级 batch。
\end{itemize}

\subsection{模型并行与张量切分}
当模型参数无法放入单卡显存时,需采用模型并行:
\begin{itemize}
  \item \textbf{张量并行(Tensor Parallelism):} 将线性层的权重按列或行切分到多张 GPU,Megatron-LM 的 Column/Row Parallel Linear 是典型实现。
  \item \textbf{序列并行:} 在自注意力计算中沿序列维度切分,减少激活冗余与通信。
  \item \textbf{ZeRO Stage-3:} 将参数、梯度和优化器状态全面分片,结合 NVMe Offload 扩展到 TB 级模型。
\end{itemize}
为了减少代价,需要平衡通信和计算;常见做法是在张量并行规模和节点内 NVLink 拓扑之间保持一致。

\subsection{流水线并行与调度}
流水线并行将模型按层划分到多个阶段,按时间片处理微批次:
\begin{itemize}
  \item \textbf{GPipe 与 1F1B(One Forward One Backward):} 通过微批次调度减少气泡时间,保持流水线饱和。
  \item \textbf{异构流水线:} 将注意力密集层与前馈层交错分配,针对 GPU/TPU 差异调整阶段长度。
  \item \textbf{检查点兼容:} 管理激活重计算与中间状态同步,确保反向传播一致性。
\end{itemize}
在大规模集群中,流水线并行常与数据、张量并行组成 3D 并行栈,DeepSpeed、Megatron-Deepspeed、Colossal-AI 等框架提供半自动化配置工具。

\section{混合精度与梯度检查点(AMP, Checkpointing)}
\subsection{自动混合精度(AMP)策略}
AMP 通过在保持数值稳定的前提下使用半精度计算,显著降低显存占用与计算成本:
\begin{itemize}
  \item \textbf{静态混合精度:} 人工指定半精度与单精度算子,例如 NVIDIA Apex 的 O1/O2 模式。
  \item \textbf{动态混合精度:} PyTorch AMP、TensorFlow Keras Mixed Precision 自动管理计算精度和缩放因子。
  \item \textbf{损失缩放:} 使用动态损失缩放避免半精度下的梯度下溢,结合 GradScaler 自动调节。
\end{itemize}
混合精度需要确保归一化、softmax、层归一等敏感算子保持更高精度,以防梯度爆炸或模式崩溃。

\subsection{梯度检查点与激活管理}
梯度检查点通过在前向传播时舍弃部分激活,在反向阶段重新计算以换取显存:
\begin{itemize}
  \item \textbf{层级检查点:} 将 Transformer Block 作为单位进行 checkpoint,减少约 50\% 激活存储。
  \item \textbf{自定义分段:} 结合长序列注意力或 MoE 层,针对高成本算子单独设置 checkpoint。
  \item \textbf{Recompute Schedules:} 自适应控制重计算层次数,在训练后期降低重计算频率以节省时间。
\end{itemize}
与 ZeRO Offload、Paged Optimizer 等方案搭配时,需要在训练框架中显式设置检查点策略,以避免重复传输。

\subsection{稳定性与验证}
混合精度和检查点策略引入新的数值风险,需要配套验证:
\begin{itemize}
  \item \textbf{数值监控:} 跟踪梯度范数、损失漂移、NaN/Inf 比率,并记录 AMP 缩放因子的曲线。
  \item \textbf{回归测试:} 在小规模数据集上与 FP32 全精度结果对比,验证收敛速度和最终性能差异。
  \item \textbf{服务一致性:} 确保训练与推理精度配置一致,避免量化或裁剪引发兼容性问题。
\end{itemize}
最终,配合自动化实验管理(Weights \& Biases、MLflow)和日志审计,可对大规模预训练进行全生命周期追踪与复现。

\section*{参考文献}
\begin{itemize}
  \item Gao et al. ``The Pile: An 800GB Dataset of Diverse Text for Language Modeling.'' arXiv, 2020.
  \item Raffel et al. ``Exploring the Limits of Transfer Learning with a Unified Text-to-Text Transformer.'' JMLR, 2020.
  \item Smith et al. ``Using DeepSpeed and Megatron to Train Megatron-Turing NLG 530B.'' arXiv, 2022.
  \item Penedo et al. ``The RefinedWeb Dataset for Falcon LLM: Outperforming Curated Corpora with Web Data, and Web Data Only.'' arXiv, 2023.
  \item Shoeybi et al. ``Megatron-LM: Training Multi-Billion Parameter Language Models Using Model Parallelism.'' arXiv, 2019.
\end{itemize}

\end{document}

