\documentclass[UTF8,zihao=-4]{ctexart}
\usepackage[a4paper,margin=2.5cm]{geometry}
\usepackage{amsmath, amssymb, amsthm}
\usepackage{bm}
\usepackage{hyperref}
\usepackage{graphicx}
\usepackage{caption}
\usepackage{listings}
\usepackage{xcolor}
\usepackage{float}
\usepackage{placeins}
\graphicspath{{figures/}}

% 代码样式
\lstdefinestyle{code}{
  basicstyle=\ttfamily\small,
  numbers=left,
  numberstyle=\tiny,
  numbersep=8pt,
  keywordstyle=\color{blue},
  commentstyle=\color{teal!70!black},
  stringstyle=\color{orange!70!black},
  showstringspaces=false,
  breaklines=true,
  frame=single,
  framerule=0.3pt,
  rulecolor=\color{black!15}
}
\lstset{style=code}

\title{大模型架构与系统优化:主流设计、归一化、激活与推理加速}
\author{}
\date{\today}

\begin{document}
\maketitle

\section{主流架构对比(GPT, LLaMA, Qwen, Mistral, Mixtral)}
\subsection{GPT 系列:解码器标准范式}
OpenAI 的 GPT-3/4 延续解码器式 Transformer,采用交替的自注意力与前馈层,使用 LayerNorm + MLP 结构,并引入多头注意力与定制的字节对编码词表。关键特征:
\begin{itemize}
  \item \textbf{高容量参数:} GPT-3(175B)使用 96 层、12288 维隐藏层;GPT-4 进一步扩展层数与多专家模块,强调对齐与多模态能力。
  \item \textbf{数据混合:} 兼顾网页、代码、对话、专业文献,配合 RLHF 与工具集成,实现泛化能力。
  \item \textbf{推理策略:} 强化上下文学习能力,使用较宽的上下文窗口和高效的 KV Cache 管理。
\end{itemize}
GPT 架构侧重稳定性与通用性,是后续模型的基准baseline。

\subsection{LLaMA 家族:轻量化与长上下文}
Meta 的 LLaMA、LLaMA2、LLaMA3 强调数据与工程效率:
\begin{itemize}
  \item \textbf{RMSNorm 替代 LayerNorm:} 提升数值稳定,同时减少归一化开销。
  \item \textbf{SwiGLU 激活:} 相较 GELU 提升表达能力,降低训练损失。
  \item \textbf{Grouped-Query Attention (GQA):} 在 LLaMA2-70B 引入组查询注意力,减少 KV Cache 存储,提升长上下文效率。
\end{itemize}
LLaMA 数据集强调开源语料 + 安全对齐,支持开源社区微调与部署。

\subsection{Qwen 系列:多语言与多模态扩展}
阿里巴巴的 Qwen-1.5/2.5 聚焦中文、英文双语兼容及工具使用:
\begin{itemize}
  \item \textbf{多语言词表:} 采用 1512M 词表,覆盖多语种并引入函数调用 token,便于工具链集成。
  \item \textbf{增强位置编码:} 使用 RoPE 外推 + 动态压缩策略支持 32K 以上上下文。
  \item \textbf{多模态模块:} 在 Qwen-VL 提供图文多模态能力,通过外部视觉编码器 + 文本解码器融合实现。
\end{itemize}
Qwen 在信息检索、代码生成等领域提供任务专项微调权重。

\subsection{Mistral 与 Mixtral:高效小模型与稀疏专家}
Mistral AI 推出的 Mistral 7B、Mixtral-8x7B 致力于推理效率:
\begin{itemize}
  \item \textbf{Sliding Window Attention:} Mistral 7B 使用滑动窗口 + short/long attention 组合,在保持性能的同时降低计算成本。
  \item \textbf{Multi-Query + Multi-Head 混合:} 减少 KV Cache,支持高吞吐生成。
  \item \textbf{Mixtral MoE:} 采用 8 专家,每次激活 top-2 专家;共享稀疏门控提升参数利用率,实现 46.7B 总参数但 12.9B 激活参数的效率优势。
\end{itemize}
Mistral 系列在开源社区广受欢迎,易于部署和推理优化。

\section{层归一化方式(LayerNorm, RMSNorm)}
\subsection{LayerNorm:经典方案}
LayerNorm 针对每个 token 的隐藏向量执行归一化:
\begin{equation}
  \mathrm{LayerNorm}(h) = \frac{h - \mu}{\sqrt{\sigma^2 + \epsilon}} \odot \gamma + \beta,
\end{equation}
其中 $\mu$ 和 $\sigma$ 为维度均值和标准差,$\gamma$、$\beta$ 为可训练缩放和偏移。优点在于稳定梯度,但存在:
\begin{itemize}
  \item 计算中需开方操作,影响性能;
  \item 对零均值假设敏感,需配合残差技巧;
  \item 在低精度或大批量训练中容易出现数值抖动。
\end{itemize}

\subsection{RMSNorm:简化与稳定}
RMSNorm 摒弃均值项,仅使用均方根归一化:
\begin{equation}
  \mathrm{RMSNorm}(h) = \frac{h}{\mathrm{rms}(h)} \odot \gamma, \qquad \mathrm{rms}(h) = \sqrt{\frac{1}{d} \sum_{i=1}^{d} h_i^2 + \epsilon}.
\end{equation}
要点:
\begin{itemize}
  \item \textbf{无偏移:} 没有 $\beta$ 项,减少参数与计算;
  \item \textbf{数值稳定:} 更适合半精度与大模型;LLaMA、Mistral 等均采用;
  \item \textbf{预归一化:} 通常在残差前应用(Pre-Norm),改善梯度流动。
\end{itemize}
实践中可根据任务需求选择,RMSNorm 在许多模型中成为默认配置。

\subsection{其他归一化变体}
还有如 DeepNorm(调整残差缩放以稳定深层网络)、ScaleNorm(固定范数)等。对于 MoE 架构,还会结合 Experts Normalization 以平衡不同专家输出。

\section{激活函数(GELU, SwiGLU)}
\subsection{GELU:高斯误差线性单元}
GELU 定义为:
\begin{equation}
  \mathrm{GELU}(x) = x \cdot \Phi(x),
\end{equation}
其中 $\Phi$ 是标准正态分布的累积分布函数。常用近似:
\begin{equation}
  \mathrm{GELU}(x) \approx 0.5 x \left(1 + \tanh\left[\sqrt{\frac{2}{\pi}} \left(x + 0.044715 x^3\right)\right]\right).
\end{equation}
特点:
\begin{itemize}
  \item 平滑开关,较 ReLU 表现更优;
  \item 在 BERT、GPT 等模型中表现良好;
  \item 计算代价适中,可使用预计算或近似加速。
\end{itemize}

\subsection{SwiGLU:门控激活}
SwiGLU 是 GLU(Gated Linear Unit)的改进,形式:
\begin{equation}
  \mathrm{SwiGLU}(x) = \mathrm{Swish}(x W_1) \odot (x W_2),
\end{equation}
其中 $\mathrm{Swish}(z) = z \cdot \sigma(z)$。特点:
\begin{itemize}
  \item 引入门控机制,增强特征选择能力;
  \item 在 LLaMA、PaLM 等模型中显著提升困惑度;
  \item 常配合 MLP 宽度放大,保持参数效率。
\end{itemize}
实际部署中需关注额外的矩阵乘法开销,可通过张量并行或混合精度处理。

\subsection{激活函数选择策略}
选择激活需考虑:
\begin{itemize}
  \item \textbf{性能与复杂度:} SwiGLU 性能高但计算多;GELU 兼顾性能与效率;
  \item \textbf{数值稳定性:} 在低精度下需评估溢出风险;
  \item \textbf{任务特性:} 生成模型通常更偏向 SwiGLU,理解模型可保留 GELU。
\end{itemize}
同时关注实验测得的困惑度、收敛速度与梯度统计。

\section{FlashAttention 与 KV Cache 优化}
\subsection{FlashAttention:显存与带宽优化}
FlashAttention 利用块状处理和寄存器重用,在不牺牲精度的前提下实现 $O(n^2)$ 注意力的高效计算。核心思想:
\begin{itemize}
  \item \textbf{分块 softmax:} 将注意力计算拆分为适配 SRAM 的小块,避免中间结果写入 HBM。
  \item \textbf{并行化:} 结合 CUDA 高效 kernel,实现 fused attention。
  \item \textbf{数值稳定:} 使用在线 softmax 技术,避免因分块带来的溢出问题。
\end{itemize}
FlashAttention v2 提升对多头、分组注意力的支持,并在 Triton 等 DSL 上实现自动调优。

\subsection{KV Cache 管理与压缩}
自回归推理中,KV Cache 存储历史 key/value,决定推理时延和显存:
\begin{itemize}
  \item \textbf{分块存储:} 按 token 维度分块写入 GPU/CPU,配合 paged attention 降低碎片化。
  \item \textbf{GQA 与 MQA:} 使用较少的 key/value 头共享多个 query,显著降低 KV Cache 大小(GQA)。
  \item \textbf{压缩技术:} 采用量化(INT8/FP8)或稀疏化策略,配合注意力重建减小误差。
\end{itemize}
在 Mistral、LLaMA 等模型中,KV Cache 优化可带来 2--4 倍吞吐提升。

\subsection{长上下文推理与检索增强}
为支持 100K+ 上下文,需要组合多项技术:
\begin{itemize}
  \item \textbf{位置编码外推:} 使用 NTK Scaling、Dynamic NTK、XPos 等方法保持高频信息。
  \item \textbf{滑动窗口与回流:} 避免存储全部历史 token,仅保留邻近窗口或使用摘要回流。
  \item \textbf{检索增强:} 将长文档切分并检索相关段落,结合 RAG 减少对 KV Cache 的依赖。
\end{itemize}
推理系统需动态调度显存与带宽,结合批处理与多租户策略实现稳定服务。

\section*{参考文献}
\begin{itemize}
  \item Vaswani et al. ``Attention Is All You Need.'' NeurIPS, 2017.
  \item Touvron et al. ``LLaMA: Open and Efficient Foundation Language Models.'' arXiv, 2023.
  \item Jiang et al. ``Mistral 7B.'' arXiv, 2023.
  \item Dao et al. ``FlashAttention: Fast and Memory-Efficient Exact Attention with IO-Awareness.'' NeurIPS, 2022.
  \item Dettmers et al. ``QLoRA: Efficient Finetuning of Quantized LLMs.'' arXiv, 2023.
\end{itemize}

\end{document}

