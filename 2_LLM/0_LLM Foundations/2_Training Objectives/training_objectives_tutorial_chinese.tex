\documentclass[UTF8,zihao=-4]{ctexart}
\usepackage[a4paper,margin=2.5cm]{geometry}
\usepackage{amsmath, amssymb, amsthm}
\usepackage{bm}
\usepackage{hyperref}
\usepackage{graphicx}
\usepackage{caption}
\usepackage{listings}
\usepackage{xcolor}
\usepackage{float}
\usepackage{placeins}
\graphicspath{{figures/}}

% 代码样式
\lstdefinestyle{code}{
  basicstyle=\ttfamily\small,
  numbers=left,
  numberstyle=\tiny,
  numbersep=8pt,
  keywordstyle=\color{blue},
  commentstyle=\color{teal!70!black},
  stringstyle=\color{orange!70!black},
  showstringspaces=false,
  breaklines=true,
  frame=single,
  framerule=0.3pt,
  rulecolor=\color{black!15}
}
\lstset{style=code}

\title{语言模型训练目标:自回归、掩码建模与优化采样实践}
\author{}
\date{\today}

\begin{document}
\maketitle

\section{自回归语言建模(Causal LM Loss)}
\subsection{目标函数与信息建模}
自回归语言模型(AR LM)假设序列生成服从链式法则,将句子 $x_{1:T}$ 的联合概率分解为条件概率乘积:
\begin{equation}
  p_\theta(x_{1:T}) = \prod_{t=1}^{T} p_\theta(x_t \mid x_{<t}).
\end{equation}
训练目标是最小化负对数似然(等价于最大化对数似然):
\begin{equation}
  \mathcal{L}_{\text{AR}}(\theta) = - \sum_{t=1}^{T} \log p_\theta(x_t \mid x_{<t}) = \sum_{t=1}^{T} \mathrm{CE}\big(\delta_{x_t}, \hat{p}_\theta(\cdot \mid x_{<t})\big),
\end{equation}
其中 $\delta_{x_t}$ 为目标词的 one-hot 分布,$\hat{p}_\theta$ 是模型在 softmax 输出空间的预测。该损失直接对模型的预测分布施加监督,使其拟合语料的条件熵结构。

AR 框架天然适配解码器式 Transformer:通过上三角掩码维持自回归约束,并借助 KV Cache 实现高效推理。具体实现中常见的两种策略是:
\begin{itemize}
  \item \textbf{Teacher Forcing:} 在训练时全部使用真实历史 token 作为条件输入,保持梯度估计无偏且便于大规模并行化。
  \item \textbf{串行采样:} 在推理时递归采样,前一步输出即为下一步的条件,模型面对暴露偏差(Exposure Bias)挑战。
\end{itemize}

\subsection{序列长度与上下文建模}
长上下文训练需要处理梯度截断、显存瓶颈和注意力复杂度。常用的配套技术包括:
\begin{itemize}
  \item \textbf{梯度累计(Gradient Accumulation):} 在较短序列上累计多次反向传播再更新参数,以近似长上下文梯度。
  \item \textbf{记忆复用(Memory Replay):} Transformer-XL、GPT-NeoX 等模型使用缓冲区拼接前一批次的隐状态,延伸有效上下文。
  \item \textbf{位置外推(RoPE、ALiBi):} 专用位置编码使模型能够在训练长度之外稳定推理。
\end{itemize}
此外,跨语种建模或多任务训练需要共享词表与归一化策略,以减小条件分布漂移。

\subsection{对比学习与正则化增强}
纯粹的语言建模损失往往侧重流畅度,难以掌握事实性与约束性。常见增强策略:
\begin{itemize}
  \item \textbf{对比约束(Contrastive Loss):} 在语言建模之外,增加噪声样本的对比项,压缩错误预测空间。
  \item \textbf{正则化项:} 标签平滑(Label Smoothing)、变长截断、随机 DropToken 等方法提高泛化,限制模型过拟合高频模式。
  \item \textbf{课程学习:} 先在短文本、低温采样上训练,再逐步扩展复杂度,帮助模型稳定收敛。
\end{itemize}

\section{掩码语言建模(Masked LM Loss)}
\subsection{双向信息建模}
掩码语言模型(MLM)通过掩蔽局部片段,让模型在双向上下文条件下预测缺失 token。对输入序列 $x_{1:T}$ 引入掩码集合 $\mathcal{M}$,训练目标为:
\begin{equation}
  \mathcal{L}_{\text{MLM}}(\theta) = - \sum_{t \in \mathcal{M}} \log p_\theta(x_t \mid x_{\setminus \mathcal{M}}).
\end{equation}
这种方式保留了自编码器(encoder-only)结构的并行优势,适合理解类任务(分类、问答、序列标注等)。然而,MLM 在推理时缺乏直接的生成机制,通常需要结合附加头或解码器。

\subsection{掩码策略设计}
掩码位置的选择直接影响语义覆盖率与训练效率。主流策略:
\begin{itemize}
  \item \textbf{随机掩码:} BERT 经典方案,以 80\% 替换为 [MASK]、10\% 替换为随机词、10\% 保留原词,提升多样性。
  \item \textbf{Whole Word Masking:} 针对中文或词粒度应用,将一个词的所有子词同时掩码,保持语义完整性。
  \item \textbf{Span Masking:} 如 SpanBERT、T5,按片段掩码以建模长距离依赖,对生成和抽取任务均有帮助。
  \item \textbf{动态掩码:} 每个 epoch 重新采样掩码位置,使模型见到的上下文组合更加多样。
\end{itemize}

\subsection{扩展任务与联合训练}
为了缓解预训练任务与下游任务的差距(pretrain–finetune mismatch),MLM 常与其他预训练目标联合:
\begin{itemize}
  \item \textbf{下一句预测(NSP)/句子顺序预测(SOP):} 强化句间关系建模,适用于段落级推理。
  \item \textbf{替换词检测(RTD):} ELECTRA 将判别任务融入预训练,以更低计算量学习高质量表示。
  \item \textbf{多任务混合:} 结合监督信号(QA、翻译、摘要),形成统一的指令或 span 级框架(如 T5 的 text-to-text)。
\end{itemize}
在多语言或跨模态场景下,MLM 可扩展为掩码语音/图像建模(HuBERT、MAE),共享统一的掩码重建范式。

\section{Tokenization(BPE, SentencePiece, tiktoken)}
\subsection{分词器设计原则}
Tokenization 决定序列长度、稀疏度和词表规模。理想的分词方案需要在以下维度取得平衡:
\begin{itemize}
  \item \textbf{覆盖度:} 词表应能重构语料,避免过多未登录词(UNK)。
  \item \textbf{压缩比:} 过大的词表增加 embedding 和 softmax 参数;过小则导致序列冗长。
  \item \textbf{跨语种适配:} 兼容多语言字符集,兼顾偏旁部首、音标等粒度。
\end{itemize}
现代 LLM 普遍采用子词(Subword)粒度,以处理开放词汇问题。

\subsection{字节对编码(BPE)}
BPE 从字符级词表出发,迭代合并出现频率最高的 token 对 $(u, v)$,将其加入词表:
\begin{enumerate}
  \item 初始化词表为字符(或字节)集合。
  \item 统计所有相邻 token 对的出现频率。
  \item 合并频率最高的对,生成新 token,并替换语料中的对应片段。
  \item 重复步骤 2--3,直到达到预设词表大小。
\end{enumerate}
优点包括可控的词表规模、良好的跨词缀泛化能力,以及对低频词的鲁棒性。BPE 适合拼写规则明确的语言,对中文等无空格语言通常先进行分词或直接基于字节对。

\subsection{SentencePiece 与 Unigram LM}
SentencePiece 提供无监督的子词建模工具,支持 BPE 与 Unigram 语言模型。Unigram LM 基于概率模型选择最优子词集合,通过 EM 算法迭代:
\begin{itemize}
  \item 赋予候选子词集合初始概率,利用前向后向算法计算句子的分词概率。
  \item 修剪低概率子词,重新归一化,直至达到目标词表规模。
\end{itemize}
SentencePiece 的关键优势是无需预分词,直接在原始字符串上构建词表,天然兼容多语言和特殊符号(emoji、标点)。Google T5、mT5、ALBERT 等均采用该方案。

\subsection{tiktoken 与现代实现}
tiktoken 是 OpenAI 针对 GPT 系列推出的高性能分词库,特点包括:
\begin{itemize}
  \item \textbf{字节级回退:} 词表缺失的 token 会自动降级为字节序列,确保编码无信息损失。
  \item \textbf{稀疏矩阵优化:} 使用 Patricia Trie 等结构实现快速匹配,显著提升编码速度。
  \item \textbf{兼容性:} 预置多种模型词表(\texttt{gpt-4}, \texttt{cl100k\_base} 等),便于推理与微调一致性。
\end{itemize}
在训练流水线中,分词器需与数据清洗、序列截断、特殊符号策略协同设计,如添加角色标记(<|assistant|>)、系统提示等。

\section{优化与采样策略(Teacher Forcing, Top-k, Top-p)}
\subsection{Teacher Forcing 与暴露偏差}
Teacher Forcing 在训练中使用真实标签作为下一步输入,使损失计算并行且梯度稳定。然而,推理阶段模型必须依赖自身预测,引入分布漂移。常见缓解方案:
\begin{itemize}
  \item \textbf{Scheduled Sampling:} 按概率混合真实 token 与模型预测,逐步过渡到生成式条件。
  \item \textbf{Professor Forcing/Adversarial Training:} 引入判别器约束训练轨迹与生成轨迹的隐状态分布一致。
  \item \textbf{强化学习微调:} 使用奖励建模(如 RLHF)在生成策略上直接优化目标。
\end{itemize}

\subsection{Top-k 采样}
Top-$k$ 采样从概率分布中筛选前 $k$ 个最有可能的 token,然后在归一化后的子分布中随机抽样:
\begin{equation}
  \mathcal{V}_k = \{x \mid p_\theta(x \mid x_{<t}) \text{ 排名前 } k\}, \quad
  p'(x) = \frac{p_\theta(x \mid x_{<t})}{\sum_{y \in \mathcal{V}_k} p_\theta(y \mid x_{<t})}.
\end{equation}
较小的 $k$ 提供更高的语言质量但降低多样性,较大的 $k$ 有助于探索,但可能产生不连贯文本。应用时常与温度(Temperature)缩放组合:
\begin{equation}
  p_\tau(x) = \frac{\exp(\log p_\theta(x)/\tau)}{\sum_y \exp(\log p_\theta(y)/\tau)}.
\end{equation}

\subsection{Top-p(Nucleus)采样}
Top-$p$ 采样根据累积概率选择最小集合 $\mathcal{V}_p$ 满足
\begin{equation}
  \mathcal{V}_p = \left\{x \mid \sum_{y \in \mathcal{V}_p} p_\theta(y \mid x_{<t}) \ge p \right\},
\end{equation}
在该集合内归一化后采样。Top-$p$ 自适应调整候选集规模,能够在不同困惑度(Perplexity)段保持稳定质量。实践中常设置 $p \in [0.8, 0.95]$,并结合最小/最大候选数约束避免极端情况。

\subsection{温度、惩罚与多样性控制}
除了 Top-$k$、Top-$p$,还可利用以下手段控制生成:
\begin{itemize}
  \item \textbf{重复惩罚(Repetition Penalty):} 对已经生成的 token 施加惩罚系数 $\gamma$,避免循环模式。
  \item \textbf{频率/存在惩罚(frequency/presence penalty):} OpenAI API 中常用,分别基于出现次数和是否出现调整 logits。
  \item \textbf{对比解码(Contrastive Decoding):} 结合小模型得分过滤低质量 token,实现更加精确的语义控制。
\end{itemize}
多样性与准确性往往相互制约,需要根据任务需求(对话、写作、代码生成等)调节参数。

\section{工程实践建议}
\begin{itemize}
  \item \textbf{数据工程:} 训练目标与分词器设计需与语料分布匹配,确保跨域泛化和毒性控制。
  \item \textbf{混合精度与优化器:} AdamW、Lion 等自适应优化器配合梯度裁剪、EMA 能够提升收敛稳定性,混合精度加速同时保持损失尺度。
  \item \textbf{对齐与评估:} 自回归/掩码模型的评估指标(困惑度、精确率、BLEU、ROUGE)应针对不同目标设计统一验证集;采样策略需在人工测评中调整,以平衡创造力和安全性。
\end{itemize}

\section*{延伸阅读}
\begin{itemize}
  \item Bengio et al. ``A Neural Probabilistic Language Model.'' Journal of Machine Learning Research, 2003.
  \item Devlin et al. ``BERT: Pre-training of Deep Bidirectional Transformers for Language Understanding.'' NAACL 2019.
  \item Radford et al. ``Language Models are Unsupervised Multitask Learners.'' OpenAI Technical Report, 2019.
  \item Holtzman et al. ``The Curious Case of Neural Text Degeneration.'' ICLR 2020.
  \item Raffel et al. ``Exploring the Limits of Transfer Learning with a Unified Text-to-Text Transformer.'' JMLR 2020.
\end{itemize}

\end{document}
