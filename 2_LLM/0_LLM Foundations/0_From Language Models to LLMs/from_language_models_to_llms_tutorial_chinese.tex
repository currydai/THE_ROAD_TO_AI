\documentclass[UTF8,zihao=-4]{ctexart}
\usepackage[a4paper,margin=2.5cm]{geometry}
\usepackage{amsmath, amssymb, amsthm}
\usepackage{bm}
\usepackage{hyperref}
\usepackage{graphicx}
\usepackage{caption}
\usepackage{listings}
\usepackage{xcolor}
\usepackage{float}
\usepackage{placeins}
\graphicspath{{figures/}}

% Code style
\lstdefinestyle{code}{
  basicstyle=\ttfamily\small,
  numbers=left,
  numberstyle=\tiny,
  numbersep=8pt,
  keywordstyle=\color{blue},
  commentstyle=\color{teal!70!black},
  stringstyle=\color{orange!70!black},
  showstringspaces=false,
  breaklines=true,
  frame=single,
  framerule=0.3pt,
  rulecolor=\color{black!15}
}
\lstset{style=code}

\title{从语言模型到大语言模型的演进}
\author{}
\date{\today}

\begin{document}
\maketitle

\section{语言模型的定义与目标函数}
语言模型通过估计文本序列 $\mathbf{x} = (x_1, \ldots, x_T)$ 的概率来度量语句的合理性,常用的因式分解为
\begin{equation}
  p_{\theta}(\mathbf{x}) = \prod_{t=1}^{T} p_{\theta}(x_t \mid x_{<t}),
\end{equation}
其中 $x_{<t}$ 表示历史上下文。最大似然训练最小化负对数似然:
\begin{equation}
  \mathcal{L}(\theta) = - \sum_{\mathbf{x} \in \mathcal{D}} \sum_{t=1}^{T} \log p_{\theta}(x_t \mid x_{<t}).
\end{equation}
困惑度(perplexity)是常见评估指标,
\begin{equation}
  \mathrm{PPL}(\mathcal{D}) = \exp\left( - \frac{1}{|\mathcal{D}|} \sum_{\mathbf{x} \in \mathcal{D}} \frac{1}{T} \log p_{\theta}(\mathbf{x}) \right),
\end{equation}
值越小说明模型越善于预测。为了防止过拟合,实践中常结合 dropout、标签平滑、梯度裁剪等正则化手段。语言建模本质上是自监督学习,仅依赖原始文本即可构建训练信号;额外的对比目标(如下一句预测、句子排序)有助于丰富语义约束。

\section{N-gram、RNN 到 Transformer 的演化}
\subsection{$n$ 元统计模型}
$n$ 元模型遵循低阶马尔可夫假设:
\begin{equation}
  p(x_t \mid x_{1:t-1}) \approx p(x_t \mid x_{t-n+1:t-1}),
\end{equation}
基于计数构造条件概率。Kneser--Ney、Good--Turing 等平滑策略可缓解稀疏,但由于上下文长度受限且参数量随 $n$ 指数增长,难以捕获长距离依赖。

\subsection{神经语言模型与循环网络}
神经语言模型引入分布式词向量和非线性组合。循环神经网络(RNN)通过隐藏状态 $\mathbf{h}_t = f_{\theta}(x_t, \mathbf{h}_{t-1})$ 汇总历史,LSTM/GRU 进一步通过门控结构(输入门、遗忘门、输出门)控制信息流:
\begin{align}
  \mathbf{i}_t &= \sigma(\mathbf{W}_i x_t + \mathbf{U}_i \mathbf{h}_{t-1}), \\
  \mathbf{f}_t &= \sigma(\mathbf{W}_f x_t + \mathbf{U}_f \mathbf{h}_{t-1}), \\
  \mathbf{c}_t &= \mathbf{f}_t \odot \mathbf{c}_{t-1} + \mathbf{i}_t \odot \tanh(\mathbf{W}_c x_t + \mathbf{U}_c \mathbf{h}_{t-1}).
\end{align}
与 $n$ 元模型相比,RNN 能捕获更长依赖,但其串行结构导致训练与推理难以并行,并且面对特别长的文本仍会退化。

\subsection{注意力机制与 Transformer}
Transformer 使用自注意力替代递归,任意两个位置之间都可以直接交互。对查询、键、值矩阵 $\mathbf{Q}$、$\mathbf{K}$、$\mathbf{V}$,缩放点积注意力为
\begin{equation}
  \mathrm{Attention}(\mathbf{Q}, \mathbf{K}, \mathbf{V}) = \mathrm{softmax}\left( \frac{\mathbf{Q}\mathbf{K}^{\top}}{\sqrt{d_k}} \right) \mathbf{V}.
\end{equation}
多头注意力、残差连接、层归一化等模块让模型既能捕获全局依赖又易于并行。相对位置编码、稀疏注意力、FlashAttention 等改进进一步拓展序列长度,为大规模预训练奠定基础。

\section{自回归(AR)与自编码(AE)的区别}
\subsection{自回归建模}
自回归模型(GPT 系列)沿时间方向逐 token 预测,用因果掩码保证仅依赖历史信息。其优点是训练稳定、推理自然,尤其适合对话、续写、代码等文本生成任务;但训练与推理之间存在暴露偏差,且无法在建模时使用右侧上下文。

\subsection{自编码建模}
自编码模型(BERT 系列)通过掩码、删除等方式破坏输入,再恢复原始 token。典型的掩码语言模型损失为
\begin{equation}
  \mathcal{L}_{\text{MLM}} = - \mathbb{E}_{\mathbf{x}, \mathbf{m}} \sum_{t \in \mathbf{m}} \log p_{\theta}(x_t \mid \mathbf{x}_{\setminus \mathbf{m}}),
\end{equation}
使模型获得双向语义表征,适合分类、阅读理解、序列标注等理解类任务。然而,直接用于流畅生成较困难,通常需引入 encoder-decoder 架构或迭代修正策略。

\subsection{混合范式}
序列到序列 Transformer、MASS、BART、T5 等方法结合 AR 与 AE 优势:编码端吸收双向上下文,解码端按自回归方式生成。Prefix LM、UniLM 通过灵活掩码在同一模型内支持理解与生成。

\section{GPT 与 BERT 的基本思想比较}
\subsection{模型结构与训练目标}
\begin{itemize}
  \item \textbf{GPT:} 采用解码器堆叠和因果掩码,目标是下一个 token 预测;训练语料覆盖海量网页、书籍和代码,强调长文本生成。
  \item \textbf{BERT:} 采用编码器堆叠和双向注意力,主要任务是掩码语言模型与句子级对比任务(NSP、SOP);聚焦于高质量语义特征。
\end{itemize}

\subsection{下游应用方式}
GPT 常通过 prompt、few-shot/in-context learning、指令微调以及 RLHF 等方式完成摘要、对话、创作等生成任务;借助工具调用、检索增强等技术,其功能不断扩展。BERT 系列通常在特定任务上添加轻量分类头或进行全量微调,用于分类、问答、实体识别、句子匹配等理解任务。

\subsection{规模化策略与演化}
GPT 路线沿着参数、数据、计算同步扩展,衍生出 GPT-3、GPT-4、PaLM、LLaMA、Mixtral 等模型,并引入 MoE、检索增强、插件生态。BERT 路线产生了 RoBERTa、DeBERTa、ELECTRA、SpanBERT 等改进版本,从目标设计、预训练语料、多模态扩展等方面持续演化。

\section{工程实践提示}
\begin{itemize}
  \item \textbf{数据治理:} 进行去重、质量过滤、多语言平衡,可提升收敛稳定性并降低隐私风险。
  \item \textbf{训练优化:} 使用混合精度、梯度检查点、ZeRO、流水线并行来控制显存与吞吐;学习率热身、余弦退火、动量校正常与之配合。
  \item \textbf{评估与安全:} 除 GLUE、SuperGLUE、MMLU、BIG-Bench 等传统指标外,还需关注幻觉、偏见、毒性等安全属性,确保可部署性。
\end{itemize}

\section*{延伸阅读}
\begin{itemize}
  \item Jurafsky \& Martin:《Speech and Language Processing》。
  \item Bengio 等:《A Neural Probabilistic Language Model》,JMLR 2003。
  \item Vaswani 等:《Attention is All You Need》,NeurIPS 2017。
  \item Devlin 等:《BERT: Pre-training of Deep Bidirectional Transformers for Language Understanding》,NAACL 2019。
  \item Kaplan 等:《Scaling Laws for Neural Language Models》,2020。
\end{itemize}

\end{document}
