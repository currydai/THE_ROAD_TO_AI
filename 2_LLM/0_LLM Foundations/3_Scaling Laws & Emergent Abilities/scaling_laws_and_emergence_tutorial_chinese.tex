\documentclass[UTF8,zihao=-4]{ctexart}
\usepackage[a4paper,margin=2.5cm]{geometry}
\usepackage{amsmath, amssymb, amsthm}
\usepackage{bm}
\usepackage{hyperref}
\usepackage{graphicx}
\usepackage{caption}
\usepackage{listings}
\usepackage{xcolor}
\usepackage{float}
\usepackage{placeins}
\graphicspath{{figures/}}

% 代码样式
\lstdefinestyle{code}{
  basicstyle=\ttfamily\small,
  numbers=left,
  numberstyle=\tiny,
  numbersep=8pt,
  keywordstyle=\color{blue},
  commentstyle=\color{teal!70!black},
  stringstyle=\color{orange!70!black},
  showstringspaces=false,
  breaklines=true,
  frame=single,
  framerule=0.3pt,
  rulecolor=\color{black!15}
}
\lstset{style=code}

\title{Scaling Laws 与涌现能力:模型规模、计算与能力边界}
\author{}
\date{\today}

\begin{document}
\maketitle

\section{模型规模、数据量、计算量的幂律关系}
\subsection{经验幂律的提出与意义}
OpenAI、DeepMind 等团队在大规模实验中发现:只要保持训练流程与数据分布一致,语言模型的损失随着模型参数 $N$、训练样本数 $D$、计算预算 $C$ 呈现稳定的幂律(Power Law)下降。以测试困惑度(Perplexity)或交叉熵损失 $\mathcal{L}$ 为例:
\begin{equation}
  \mathcal{L}(N) \approx A_N N^{-\alpha} + B, \quad
  \mathcal{L}(D) \approx A_D D^{-\beta} + B, \quad
  \mathcal{L}(C) \approx A_C C^{-\gamma} + B,
\end{equation}
其中 $A_\ast, B, \alpha, \beta, \gamma$ 为常数。幂律体现了“规模=性能”的基本趋势,是分配计算资源、确定预训练策略的理论依据。

幂律关系的稳定性带来两个重要启示:
\begin{itemize}
  \item \textbf{可预测性:} 通过小规模实验拟合幂律指数,可外推更大规模模型的期望损失与所需计算量。
  \item \textbf{资源最优分配:} 在总计算预算固定情况下,选择合适的模型大小与训练 tokens 数,可最大化性能。
\end{itemize}

\subsection{Pythia/GPT 系列实验趋势}
Pythia、GPT-3 等族谱验证了上述幂律,指出当数据量不足时模型进入“数据受限”阶段,参数继续增大收益减弱;反之,模型容量不足会成为“模型受限”瓶颈。图~\ref{fig:power_law_placeholder_cn} 以示意方式呈现不同受限区域。

\begin{figure}[H]
  \centering
  \fbox{\begin{minipage}{0.85\textwidth}
    \centering
    \textbf{示意图:}左半区为数据受限,右半区为模型受限;对角线附近代表计算最优分配。
  \end{minipage}}
  \caption{幂律缩放示意:在双对数坐标下,损失沿幂律衰减。}
  \label{fig:power_law_placeholder_cn}
\end{figure}
\FloatBarrier

\subsection{计算预算与训练效率}
综合考虑模型规模与数据量,可将训练 FLOPs 近似表示为
\begin{equation}
  C \approx 6 N D,
\end{equation}
其中常数系数受网络深度、序列长度、并行策略影响。工程实践中常见的两种策略:
\begin{itemize}
  \item \textbf{参数翻倍优先:} 在推理为主的场景中倾向于增大模型,使困惑度下降带来更强能力。
  \item \textbf{数据扩展优先:} 在线上持续学习场景,通过增加高质量数据与训练步数维持性能。
\end{itemize}
此外,复用冻结底座模型并在增量数据上微调,可显著降低额外计算成本。

\section{Chinchilla Scaling Law}
\subsection{Chinchilla 论文核心结论}
DeepMind 在 2022 年提出的 Chinchilla Scaling Law 对传统幂律做出修正:在固定计算预算下,最优策略是让训练 token 数 $D$ 与模型参数 $N$ 近似相等($D \propto N$),而非 GPT-3 式的“参数远大于训练 token”。论文给出的最优损失公式:
\begin{equation}
  \mathcal{L}(N, D) = \bigg(\frac{N}{N_0}\bigg)^{-0.34} + \bigg(\frac{D}{D_0}\bigg)^{-0.28} + \mathcal{L}_\infty,
\end{equation}
并通过实验验证当 $D \approx 20 N$ (以 token 计算) 时性能最佳。Chinchilla 模型以 70B 参数在近 1.4T token 上训练,性能超越更大但训练不足的 Gopher。

\subsection{对资源规划的影响}
Chinchilla Scaling Law 带来以下实操建议:
\begin{itemize}
  \item \textbf{优先增加数据:} 如果过去的训练只覆盖 $D \ll N$,应优先扩充高质量语料而不是盲目增加参数。
  \item \textbf{高效再训练:} 将旧模型重训一次可能比增加参数或长时间微调更划算。
  \item \textbf{数据质量重要性:} 数据集多样性、去重、过滤对最终性能影响更大,尤其在需要海量 token 的 regime。
\end{itemize}

Chinchilla 还强调了推理成本:较小但训练充分的模型不仅性能更好,而且推理延迟更低,有利于部署。

\subsection{算法与硬件协同}
要满足 Chinchilla 的 Data-Optimal 需求,需要在工程上实现高速数据流与分布式训练:
\begin{itemize}
  \item \textbf{流水线并行 + 张量并行:} 平衡参数与数据维度的通信负载。
  \item \textbf{异构存储:} 利用 NVMe 缓存与分布式文件系统保证 token 读写速度。
  \item \textbf{数据增广与清洗流水线:} 自适应采样、重复过滤、语种配比等策略避免冗余。
\end{itemize}

\section{“涌现能力”案例:推理、记忆与组合泛化}
\subsection{涌现能力的定义与争议}
“涌现能力”(Emergent Abilities)指模型在扩展至某一规模阈值后,突然在特定任务上表现出质的提升,如复杂推理、多步规划等。实践中常通过阈值效应或非线性指标观察:
\begin{equation}
  \text{Ability} \approx \begin{cases}
    \text{baseline}, & N < N^\star \\
    \text{rapid improvement}, & N \ge N^\star
  \end{cases}
\end{equation}
然而,最近研究指出涌现可能是指标选择导致的“测量幻觉”。若采用平滑的损失函数(如对数失真)衡量能力,增长通常仍呈幂律连续变化。因此,需要精心设计评价指标,避免将离散准确率的跳变误认为真正的涌现。

\subsection{推理(Reasoning)能力}
案例包括多步算术、符号推理、链式思考(Chain-of-Thought, CoT):
\begin{itemize}
  \item \textbf{算术推理:} 在 GSM8K、SVAMP 等数据集上,大模型通过 CoT 提示实现接近小学甚至初中水平的分步解题。
  \item \textbf{程序辅助推理:} 结合外部工具(Python 解释器)执行算术,可显著提升正确率,显示出“工具使用”能力。
  \item \textbf{多跳问答:} HotpotQA、StrategyQA 等任务要求跨段落推演,大模型在上下文检索与逻辑连续性方面表现显著优于小模型。
\end{itemize}
推理能力常通过自一致性(self-consistency)采样增强:对同一问题生成多条思路,汇聚最常见结论,从而平滑不可控的输出。

\subsection{记忆(Memory)与长期上下文}
大型模型能够在极长上下文(>32k token)中准确检索与引用信息,体现出类记忆能力:
\begin{itemize}
  \item \textbf{内隐记忆:} 模型从预训练数据中存储事实,如 API 调用、歌曲歌词。Chinchilla 最优策略下的充分训练有助于巩固这些记忆。
  \item \textbf{外显记忆:} 通过检索增强(RAG)、工具调用、插件系统,在推理阶段查询数据库,实现动态记忆。
  \item \textbf{长上下文测试:} Needle-in-a-Haystack、Long Range Arena 等基准衡量模型在海量文本中定位关键信息的能力。
\end{itemize}
记忆表现与分词策略、KV Cache 设计相关;缓存压缩、相对位置编码、分块注意力等技术对长上下文至关重要。

\subsection{组合泛化(Compositional Generalization)}
组合泛化指模型在训练未覆盖的组合上仍能正确执行任务。例如,在 SCAN、COGS、PCFG 等数据集上,模型需组合已学动作/语法以完成新任务。大型语言模型的表现要点:
\begin{itemize}
  \item \textbf{少样本学习:} 通过 prompt 中的示例组合出新的逻辑结构,展示“类系统 2”推理迹象。
  \item \textbf{思维树(Tree-of-Thought, ToT):} 在搜索树中探索不同子计划,再合成最终答案,提高解题稳健性。
  \item \textbf{程序合成:} Codex、AlphaCode 等系统在编程竞赛题上展示组合泛化能力,能够综合多个子函数与数据结构。
\end{itemize}

\subsection{评估与监测工具}
为了追踪涌现能力,需要构建细粒度的评估矩阵:
\begin{itemize}
  \item \textbf{Benchmark 套件:} BIG-Bench、MMLU、AGIEval 汇集语言、逻辑、专业知识等多维任务,观察性能曲线。
  \item \textbf{能力触发实验:} 比较不同规模模型在特定任务上的“阈值点”,并使用平滑指标验证增长是否真正非线性。
  \item \textbf{任务标签与元数据:} 为每个任务记录思维类型、步骤长度、外部工具需求,以分析能力涌现的模式。
\end{itemize}

\section{附录:幂律拟合示例代码}
\begin{lstlisting}[language=Python,caption={根据实验点拟合规模幂律指数的示例},label={lst:powerlaw_cn}]
import numpy as np
from scipy import stats

params = np.array([1e8, 3e8, 1e9, 3e9])
loss = np.array([3.1, 2.7, 2.4, 2.2])  # 以困惑度或交叉熵为例

log_params = np.log10(params)
slope, intercept, r, p, stderr = stats.linregress(log_params, loss)

alpha = -slope
print(f"拟合到的幂律指数 alpha ≈ {alpha:.3f}")
print(f"模型损失近似 L(N) ≈ 10^{intercept:.3f} * N^-{alpha:.3f}")
\end{lstlisting}

\section{工程实践建议}
\begin{itemize}
  \item \textbf{实验规划:} 先在小规模上采样多个组合 $(N, D)$,拟合幂律后再决定大规模训练配置。
  \item \textbf{数据治理:} 针对目标能力设计数据配方(math/code/多语种),并持续监测能力曲线。
  \item \textbf{对齐策略:} 对于推理、记忆和组合泛化任务,在基础模型上叠加指令微调、CoT 提示、工具链使用以稳定能力输出。
\end{itemize}

\section*{延伸阅读}
\begin{itemize}
  \item Kaplan et al. ``Scaling Laws for Neural Language Models.'' arXiv:2001.08361, 2020.
  \item Hoffmann et al. ``Training Compute-Optimal Large Language Models.'' arXiv:2203.15556, 2022.
  \item Wei et al. ``Emergent Abilities of Large Language Models.'' arXiv:2206.07682, 2022.
  \item Schaeffer et al. ``Are Emergent Abilities of Large Language Models a Mirage?'' arXiv:2304.15004, 2023.
  \item Srivastava et al. ``Beyond the Imitation Game: Quantifying and Extrapolating the Capabilities of Language Models.'' BIG-Bench, 2022.
\end{itemize}

\end{document}
